%--------1---------2---------3---------4---------5---------6---------7
%
% Competition paper
% SBST 2013
% Page limit: 4 pages
%

%\documentclass[10pt, conference, compsocconf]{IEEEtran}
\documentclass[10pt,conference,compsocconf]{IEEEtran}

%\documentclass[times, 10pt,twocolumn]{article} 
%\usepackage{latex8}
%\usepackage{times}
\usepackage{amsfonts}
\usepackage[latin1]{inputenc}
\usepackage[english]{babel}
\usepackage{listings}
\usepackage{algorithmic}
\usepackage{float}
\usepackage{cite}
\usepackage{graphicx}
\usepackage{booktabs}
\usepackage{subfigure}
\usepackage{hyperref}
\usepackage{color}
\usepackage[usenames,dvipsnames,table]{xcolor}
\usepackage{soul}
\usepackage{xspace}
\usepackage{boxedminipage}
\usepackage{alltt}
\usepackage{multirow}
\usepackage{paralist}
\usepackage{amsmath}



\floatstyle{ruled}
\newfloat{algorithm}{tbp}{loa}
\floatname{algorithm}{Algorithm}

\newtheorem{definition}{Definition}


 %krams hinter fontadjust ist neu
  \definecolor{lightgrey}{rgb}{0.90,0.90,0.90}
\lstset{escapeinside={(*}{*)}}
  \lstloadlanguages{java}
 \lstdefinelanguage{pseudocode}
  {morekeywords={if, else, initialize, return, for, each, in, global, new}
   }
  \lstset{
    tabsize=2,
    mathescape=true,
    escapeinside={(*}{*)},
    captionpos=t,
    framerule=0pt,
    backgroundcolor=\color{lightgrey},
    basicstyle=\footnotesize\ttfamily,
    keywordstyle=\footnotesize\bfseries,
    numbers=none,
    numberstyle=\tiny,
    numbersep=1pt,
    fontadjust,
    breaklines=true,
    breakatwhitespace=false
  }    
      

\hypersetup{
colorlinks=true,
urlcolor=rltblue,
linkcolor=rltred,
citecolor=rltgreen,
bookmarksnumbered=true,
pdftitle={EvoSuite at the SBST 2016 Tool Competition},
pdfauthor={Gordon Fraser and Andrea Arcuri},
pdfsubject={Test case generation},
pdfkeywords={Test case generation, unit testing, test
  oracles, assertions, search based testing}
}

\definecolor{rltred}{rgb}{0.5,0,0}
\definecolor{rltgreen}{rgb}{0,0.5,0}
\definecolor{rltblue}{rgb}{0,0,0.5}
\definecolor{ScarletRed}{rgb}{0.80,0.00,0.00}



% in draft mode we put \remarks into the margins and do other stuff
% set to \draftfalse for 
\newif\ifdraft
\draftfalse

\ifdraft
	\marginparwidth=1.3cm
	\marginparsep=5pt
	\newcommand\remark[1]{%
		\mymarginpar{\raggedright\hbadness=10000\tiny\it #1\par}}
\else
	\newcommand\remark[1]	{}
\fi

\ifdraft
	\overfullrule3pt
\fi    

% We use \FIXME for located problems (``defect'')
\newcommand{\FIXME}[1]{\remark{FIXME: #1}}
\newcommand\parremark[1]	{\par\textbf{REMARK:} #1\par}

\newcommand{\gordon}[1]{\textcolor{blue}{\sf\small\textbf{Gordon:} #1}}
\newcommand{\andrea}[1]{\textcolor{ScarletRed}{\sf\small\textbf{Andrea:} #1}}

% \mathid is used to denote identifiers and slots in formulas
\newcommand{\mathid}[1]{\text{\rmfamily\textit{#1}}}

% But usually, we shall use \|name| instead.
\def\|#1|{\mathid{#1}}

% \codeid is used to denote computer code identifiers
\newcommand{\codeid}[1]{\texttt{#1}}

% But usually, we shall use \<name> instead.
\def\<#1>{\codeid{#1}}

% Our results
\newenvironment{result}%
{\smallskip
\noindent
\let\emph=\textbf
\begin{boxedminipage}{\columnwidth}\begin{center}\em}%
{\end{center}\end{boxedminipage}%
\smallskip
}

\newcommand{\JodaTime}{Joda-Time\xspace}  % That's how they write themselves -- AZ

\newcommand{\EVOSUITE}{{\sc EvoSuite}\xspace}
\newcommand{\MUTEST}{{\sc $\mu$Test}\xspace}
\newcommand{\CS}{{\sc SF100}\xspace}

% TODO marker
\newcommand{\TODO}[1]{\sethlcolor{yellow}\textbf{\textcolor{ScarletRed}{\hl{TODO: #1}}}\xspace}


\DeclareMathSymbol{,}{\mathpunct}{letters}{"3B}
\DeclareMathSymbol{,}{\mathord}{letters}{"3B}
\DeclareMathSymbol{\decimal}{\mathord}{letters}{"3A}
%%%"

%\documentstyle[times,art10,twocolumn,latex8]{article}

%------------------------------------------------------------------------- 
% take the % away on next line to produce the final camera-ready version 
%\pagestyle{empty}

%------------------------------------------------------------------------- 
\begin{document}

%\title{Unit Testing Tool competition: Results for EvoSuite}
\title{EvoSuite at the SBST 2016 Tool Competition}

\author{
\IEEEauthorblockN{Gordon Fraser}
\IEEEauthorblockA{University of Sheffield\\
Sheffield, UK}\\
%gordon.fraser@sheffield.ac.uk}\\
\and
\IEEEauthorblockN{Andrea Arcuri}
\IEEEauthorblockA{Scienta, Norway\\
and University of Luxembourg}\\
%arcuri@simula.no}
%\and
%\IEEEauthorblockN{Jeremias R\"{o}{\ss}ler}
%\IEEEauthorblockA{Saarland University -- Computer Science\\
%Saarbr\"ucken, Germany\\
%roessler@cs.uni-saarland.de}
}

\maketitle
%\thispagestyle{empty}

\begin{abstract}
  \EVOSUITE is a search-based tool that automatically
  generates unit tests for Java code.  This paper summarizes the
  results and experiences of \EVOSUITE's participation at the fourth
  unit testing competition at SBST 2016. 
\end{abstract}

\begin{IEEEkeywords}
  test case generation; search-based testing; testing classes;
  search-based software engineering
\end{IEEEkeywords}


%------------------------------------------------------------------------- 
\section{Introduction}

This paper describes the results of applying the \EVOSUITE test
generation tool~\cite{FrA11c} to the benchmark used in the tool
competition at the International Workshop on Search-Based Software
Testing (SBST) 2016.  Details about the competition and the benchmark
can be found in []. 
In this competition, \EVOSUITE ranked [] with a score of [].

%------------------------------------------------------------------------- 
\section{About \EVOSUITE}


\begin{table}[!h]
\renewcommand{\arraystretch}{1.3}
\caption{Classification of the \EVOSUITE unit test generation tool.}\label{tool-description}
\begin{tabular}{|l|p{5cm}|}
  \hline
  \multicolumn{2}{|l|}{Prerequisites} \\
  \hline
  Static or dynamic &  Dynamic testing at the Java class level\\
  Software Type &  Java classes\\
  Lifecycle phase&  Unit testing for Java programs\\
  Environment&  All Java development environments \\
  Knowledge required & JUnit unit testing for Java\\
  Experience required &  Basic unit testing knowledge\\
 \hline
  \multicolumn{2}{|l|}{Input and Output of the tool} \\
  \hline
 Input & Bytecode of the target class and dependencies \\
\hline
Output&  JUnit test cases (version 3 or 4)\\
 
  \hline
  \multicolumn{2}{|l|}{Operation} \\
  \hline
  Interaction &  Through the command line, and plugins for IntelliJ, Maven and Eclipse\\
  User guidance &  manual verification of assertions for functional faults\\
  Source of information &  http://www.evosuite.org \\
  Maturity&  Mature research prototype, under development\\
  Technology behind the tool & Search-based testing / whole test suite generation\\
\hline
  \multicolumn{2}{|l|}{Obtaining the tool and information} \\
  \hline
License & Lesser GPL V.3\\
Cost & Open source\\
Support & None \\
\hline
\hline
  \multicolumn{2}{|l|}{Does there exist empirical evidence about} \\
  \hline
  Effectiveness and Scalability & See~\cite{GoA_TSE12,fraser2014large}. \\
%Completeness & \\
%Effectiveness & \\
%Efficiency & \\
%Defect types & \\
%Scalability & \\
%Comprehensibility & \\
%Learnability & \\
%Subjective satisfaction & \\
%Other & \\
\hline
\end{tabular}\vspace{-1em}
\end{table}


\EVOSUITE~\cite{FrA11c,GoA_TSE12} automatically generates test suites
for Java classes, targeting branch coverage and other coverage
criteria (e.g., mutation testing~\cite{emse14_mutation}). \EVOSUITE
works at the Java bytecode level, i.e., it does not require source
code. It is fully automated and requires no manually written test
drivers or parameterized unit tests.  For example, when \EVOSUITE is
used from its Eclipse and IntelliJ plugins, a user
just needs to select a class, and tests are generated with a mouse-click.

\EVOSUITE has been evaluated on millions of lines of Java
code~\cite{fraser2014large}, both open-source code and close-source
code provided by one of our industrial partners.  In the first two
editions of the unit testing tool competition, \EVOSUITE ranked
first~\cite{evosuiteAtSbst2013,evosuiteAtFittest2013}, whereas it
ranked second in the third one [].


\EVOSUITE uses an evolutionary approach to derive these test suites: A
genetic algorithm evolves candidate individuals (chromosomes) using
operators inspired by natural evolution (e.g., selection, crossover
and mutation), such that iteratively better solutions with respect to
the optimization target (e.g., branch coverage) are produced.  For
details on this test generation approach we refer to~\cite{GoA_TSE12}.
%To improve performance further, we are investigating several
%extensions to \EVOSUITE. 
%For example, \EVOSUITE can employ dynamic
%symbolic execution~\cite{evoISSRE113} 
%and memetic algorithms~\cite{fraser2014memetic}
%to handle the cases in which our genetic algorithm may struggle. 


As the generated unit tests are meant for human
consumption~\cite{fraser2013does}, \EVOSUITE applies various
post-processing steps to improve readability (e.g., minimising) and
adds test assertions that capture the current behavior of the tested
classes. To select the most effective assertions, \EVOSUITE uses
mutation analysis~\cite{10.1109/TSE.2011.93}.  \EVOSUITE can also be
used to automatically find faults such as undeclared thrown exceptions
and broken code contracts~\cite{emse13_oracle}.  For more details on
the tool and its abilities we refer to~\cite{FrA11c}, and for more
implementation details we refer to~\cite{FrA13a}.

%------------------------------------------------------------------------- 
\section{Competition Setup}

\EVOSUITE can be configured to target different coverage criteria. The
fitness function to drive the genetic algorithm was based on a
combination of several criteria~\cite{rojas2015combining} (e.g., line
coverage, branch coverage, branch coverage by direct method
invocation, weak mutation testing, output coverage, exception
coverage). \EVOSUITE now by default uses an archive of
solutions~\cite{emse_archive}, which means that throughout the search,
whenever a new coverage goal is satisfied, the corresponding test is
stored in the archive, and this goal is no longer targeted by the
fitness function. We enabled the post-processing step of test
minimization, but to reduce the time spent we included all assertions
rather than filtering them with mutation
analysis~\cite{10.1109/TSE.2011.93}. The use of all assertions has
effects on readability and the chances of obtaining flaky
tests. However, as readability is not measured by the SBST contest
metric, and many of the improvements to \EVOSUITE since the last
competition target flaky tests we deemed this not a problem.

In contrast to previous instances of the competition, the test
generation tools this time received a time budget as input, and then
had to generate tests within that time. \EVOSUITE uses a combination
of different timeouts for its individual phases (e.g., initialization,
search, minimization, assertion generation, compilation check, removal
of flaky tests), which created the challenge of distributing the
overall budget onto these phases. We used a simple approach where 50\%
of the time was allocated to the search, whereas the other half of the
time was distributed equally to the remaining phases. If any of the
phases used more time than allocated, which can for example happen if
test executions take long or lead to timeouts, then phases for which
there is no time left are skipped. For example, if there is no time
left for minimization, then the raw test suite as generated by the
search is returned.

%------------------------------------------------------------------------- 
\section{Benchmark results}

\begin{table*}[t]
  \centering
  \caption{\label{table:results}Detailed results of \EVOSUITE on the SBST benchmark classes.}
\resizebox{0.9\textwidth}{!}{  
\begin{tabular}{ ll rr rr rr}\toprule 
 \multirow{2}{1in}{Benchmark} & \multirow{2}{1in}{Java Class} &  \multicolumn{2}{c}{Line Coverage} &  \multicolumn{2}{c}{Branch Coverage} &  \multicolumn{2}{c}{Mutation Score}\\\cmidrule(lr){3-4}\cmidrule(lr){5-6} \cmidrule(lr){7-8}
 & & 60s & 180s & 60s & 180s & 60s & 180s \\ 
\midrule 
FESCAR-10  &  com.alibaba.fescar.core.model.BranchType & 80.0\% & 90.0\% & 80.0\% & 90.0\% & 80.0\% & 90.0\%\\ 
FESCAR-12  &  com.alibaba.fescar.core.rpc.netty.RpcServerHandler & 100.0\% & 100.0\% & 87.5\% & 87.5\% & 100.0\% & 100.0\%\\ 
FESCAR-13  &  com.alibaba.fescar.core.exception.TransactionExceptionCode & 100.0\% & 100.0\% & 100.0\% & 100.0\% & 100.0\% & 100.0\%\\ 
FESCAR-15  &  com.alibaba.fescar.core.rpc.netty.RpcServer & 0.8\% & 0.7\% & \cellcolor{light-gray} \textcolor{black}{0.0\%} & \cellcolor{light-gray} \textcolor{black}{0.0\%} & \cellcolor{light-gray} \textcolor{black}{0.0\%} & \cellcolor{light-gray} \textcolor{black}{0.0\%}\\ 
FESCAR-17  &  com.alibaba.fescar.core.protocol.transaction.GlobalBeginResponse & 99.4\% & 99.4\% & 100.0\% & 100.0\% & 90.0\% & 90.0\%\\ 
FESCAR-2  &  com.alibaba.fescar.core.service.ServiceManagerStaticConfigImpl & 20.5\% & 25.8\% & \cellcolor{light-gray} \textcolor{black}{0.0\%} & \cellcolor{light-gray} \textcolor{black}{0.0\%} & \cellcolor{light-gray} \textcolor{black}{0.0\%} & \cellcolor{light-gray} \textcolor{black}{0.0\%}\\ 
FESCAR-23  &  com.alibaba.fescar.core.protocol.MergeResultMessage & 90.5\% & 60.5\% & 76.4\% & 50.0\% & \cellcolor{light-gray} \textcolor{black}{0.0\%} & \cellcolor{light-gray} \textcolor{black}{0.0\%}\\ 
FESCAR-25  &  com.alibaba.fescar.core.rpc.netty.RmMessageListener & 46.9\% & 37.5\% & 62.5\% & 48.8\% & 22.2\% & 17.8\%\\ 
FESCAR-28  &  com.alibaba.fescar.core.rpc.ClientType & 90.0\% & 100.0\% & 90.0\% & 100.0\% & 90.0\% & 100.0\%\\ 
FESCAR-32  &  com.alibaba.fescar.core.protocol.transaction.BranchRegisterRequest & 97.7\% & 89.2\% & 94.4\% & 87.5\% & 95.2\% & 78.3\%\\ 
FESCAR-33  &  com.alibaba.fescar.core.model.GlobalStatus & 100.0\% & 100.0\% & 100.0\% & 100.0\% & 100.0\% & 100.0\%\\ 
FESCAR-34  &  com.alibaba.fescar.core.protocol.ResultCode & 90.0\% & 100.0\% & 90.0\% & 100.0\% & 90.0\% & 100.0\%\\ 
FESCAR-37  &  com.alibaba.fescar.core.rpc.RpcContext & 92.4\% & 94.6\% & 86.8\% & 91.2\% & \cellcolor{light-gray} \textcolor{black}{0.0\%} & \cellcolor{light-gray} \textcolor{black}{0.0\%}\\ 
FESCAR-41  &  com.alibaba.fescar.core.rpc.netty.RmRpcClient & 1.7\% & 1.7\% & 2.0\% & 2.0\% & \cellcolor{light-gray} \textcolor{black}{0.0\%} & 2.4\%\\ 
FESCAR-42  &  com.alibaba.fescar.core.rpc.DefaultServerMessageListenerImpl & 24.3\% & 42.6\% & 11.8\% & 27.1\% & 12.1\% & 25.4\%\\ 
FESCAR-5  &  com.alibaba.fescar.core.protocol.MessageFuture & 98.6\% & 99.1\% & 96.0\% & 98.0\% & 99.2\% & 100.0\%\\ 
FESCAR-6  &  com.alibaba.fescar.core.rpc.netty.TmRpcClient & 3.4\% & 3.4\% & 2.7\% & 2.7\% & \cellcolor{light-gray} \textcolor{black}{0.0\%} & 2.7\%\\ 
FESCAR-7  &  com.alibaba.fescar.core.rpc.netty.MessageCodecHandler & 76.1\% & 78.2\% & 73.3\% & 77.2\% & \cellcolor{light-gray} \textcolor{black}{0.0\%} & \cellcolor{light-gray} \textcolor{black}{0.0\%}\\ 
FESCAR-8  &  com.alibaba.fescar.core.rpc.netty.NettyPoolableFactory & 57.3\% & 62.0\% & 50.8\% & 57.5\% & \cellcolor{light-gray} \textcolor{black}{0.0\%} & \cellcolor{light-gray} \textcolor{black}{0.0\%}\\ 
FESCAR-9  &  com.alibaba.fescar.core.protocol.transaction.GlobalBeginRequest & 99.0\% & 98.3\% & 100.0\% & 100.0\% & 99.1\% & 98.2\%\\ 
GUAVA-102  &  com.google.common.collect.LinkedListMultimap & 29.4\% & 32.3\% & 12.9\% & 11.6\% & 19.2\% & 14.8\%\\ 
GUAVA-110  &  com.google.common.collect.LexicographicalOrdering & 3.0\% & 22.2\% & \cellcolor{light-gray} \textcolor{black}{0.0\%} & 7.5\% & 0.6\% & 15.0\%\\ 
GUAVA-128  &  com.google.common.base.Throwables & 75.1\% & 25.0\% & 75.8\% & 25.3\% & 81.0\% & 26.8\%\\ 
GUAVA-129  &  com.google.common.collect.SparseImmutableTable & 31.9\% & 35.8\% & 37.5\% & 42.5\% & 35.0\% & 43.8\%\\ 
GUAVA-159  &  com.google.common.primitives.ParseRequest & 100.0\% & 100.0\% & 100.0\% & 100.0\% & 50.0\% & 50.0\%\\ 
GUAVA-169  &  com.google.common.math.LongMath & 96.2\% & 86.7\% & 94.2\% & 85.3\% & 99.2\% & 89.3\%\\ 
GUAVA-177  &  com.google.common.primitives.Doubles & 98.7\% & 98.5\% & 99.3\% & 99.3\% & 100.0\% & 100.0\%\\ 
GUAVA-181  &  com.google.common.primitives.SignedBytes & 100.0\% & 100.0\% & 100.0\% & 100.0\% & 100.0\% & 100.0\%\\ 
GUAVA-184  &  com.google.thirdparty.publicsuffix.PublicSuffixType & 100.0\% & 100.0\% & 100.0\% & 100.0\% & 100.0\% & 100.0\%\\ 
GUAVA-196  &  com.google.common.io.Closeables & 71.5\% & 70.0\% & 77.5\% & 75.0\% & 88.0\% & 88.0\%\\ 
GUAVA-2  &  com.google.common.collect.MinMaxPriorityQueue & 13.9\% & 22.5\% & 6.4\% & 11.1\% & 16.5\% & 19.2\%\\ 
GUAVA-206  &  com.google.common.collect.ImmutableEnumSet & 25.4\% & 26.1\% & 23.6\% & 24.5\% & 7.1\% & 7.6\%\\ 
GUAVA-212  &  com.google.common.net.MediaType & 92.6\% & 94.3\% & 77.6\% & 83.0\% & \cellcolor{light-gray} \textcolor{black}{0.0\%} & \cellcolor{light-gray} \textcolor{black}{0.0\%}\\ 
GUAVA-22  &  com.google.common.graph.Graphs & 53.9\% & 49.7\% & 51.8\% & 47.3\% & \cellcolor{light-gray} \textcolor{black}{0.0\%} & \cellcolor{light-gray} \textcolor{black}{0.0\%}\\ 
GUAVA-224  &  com.google.common.primitives.UnsignedLongs & 99.3\% & 89.6\% & 100.0\% & 90.0\% & 100.0\% & 90.0\%\\ 
GUAVA-240  &  com.google.common.collect.FilteredMultimapValues & 12.3\% & 22.7\% & \cellcolor{light-gray} \textcolor{black}{0.0\%} & 5.0\% & \cellcolor{light-gray} \textcolor{black}{0.0\%} & \cellcolor{light-gray} \textcolor{black}{0.0\%}\\ 
GUAVA-39  &  com.google.common.collect.TreeMultiset & 30.2\% & 43.1\% & 18.6\% & 27.9\% & 19.5\% & 31.3\%\\ 
GUAVA-47  &  com.google.common.collect.FilteredEntryMultimap & 2.6\% & 11.3\% & \cellcolor{light-gray} \textcolor{black}{0.0\%} & 0.7\% & \cellcolor{light-gray} \textcolor{black}{0.0\%} & 0.4\%\\ 
GUAVA-90  &  com.google.common.io.FileBackedOutputStream & 98.9\% & 89.6\% & 98.1\% & 90.0\% & 98.0\% & 89.3\%\\ 
GUAVA-95  &  com.google.common.collect.ComparatorOrdering & 27.5\% & 51.7\% & 12.5\% & 30.0\% & 18.8\% & 31.2\%\\ 
PDFBOX-117  &  org.apache.pdfbox.filter.Predictor & 89.0\% & 93.5\% & 83.9\% & 91.0\% & \cellcolor{light-gray} \textcolor{black}{0.0\%} & 28.6\%\\ 
PDFBOX-127  &  org.apache.pdfbox.pdfparser.PDFObjectStreamParser & 57.5\% & 65.6\% & 37.1\% & 43.6\% & 44.4\% & 50.6\%\\ 
PDFBOX-130  &  org.apache.pdfbox.pdmodel.interactive.digitalsignature.visible.PDVisibleSignDesigner & 7.1\% & 14.3\% & 1.7\% & 1.7\% & 1.5\% & 2.5\%\\ 
PDFBOX-157  &  org.apache.pdfbox.pdmodel.font.PDType1Font & 2.1\% & \cellcolor{light-gray} \textcolor{black}{0.0\%} & 0.4\% & \cellcolor{light-gray} \textcolor{black}{0.0\%} & \cellcolor{light-gray} \textcolor{black}{0.0\%} & \cellcolor{light-gray} \textcolor{black}{0.0\%}\\ 
PDFBOX-198  &  org.apache.pdfbox.pdmodel.fdf.FDFAnnotationLine & 66.4\% & 66.5\% & 32.4\% & 32.7\% & 5.5\% & \cellcolor{light-gray} \textcolor{black}{0.0\%}\\ 
PDFBOX-214  &  org.apache.pdfbox.pdfparser.EndstreamOutputStream & 99.5\% & 90.0\% & 99.2\% & 90.0\% & 48.0\% & 40.0\%\\ 
PDFBOX-22  &  org.apache.pdfbox.pdmodel.fdf.FDFAnnotationCaret & 63.9\% & 63.9\% & 64.3\% & 64.3\% & 10.5\% & 31.4\%\\ 
PDFBOX-220  &  org.apache.pdfbox.filter.JPXFilter & 32.7\% & 32.7\% & 7.7\% & 7.3\% & \cellcolor{light-gray} \textcolor{black}{0.0\%} & \cellcolor{light-gray} \textcolor{black}{0.0\%}\\ 
PDFBOX-229  &  org.apache.pdfbox.util.XMLUtil & 62.4\% & 69.6\% & 52.5\% & 60.0\% & 10.7\% & 13.6\%\\ 
PDFBOX-234  &  org.apache.pdfbox.pdmodel.interactive.action.PDActionSound & 97.7\% & 96.7\% & 88.9\% & 87.8\% & \cellcolor{light-gray} \textcolor{black}{0.0\%} & 20.0\%\\ 
PDFBOX-235  &  org.apache.pdfbox.pdmodel.font.PDTrueTypeFontEmbedder & \cellcolor{light-gray} \textcolor{black}{0.0\%} & \cellcolor{light-gray} \textcolor{black}{0.0\%} & \cellcolor{light-gray} \textcolor{black}{0.0\%} & \cellcolor{light-gray} \textcolor{black}{0.0\%} & \cellcolor{light-gray} \textcolor{black}{0.0\%} & \cellcolor{light-gray} \textcolor{black}{0.0\%}\\ 
PDFBOX-26  &  org.apache.pdfbox.pdmodel.encryption.SecurityProvider & 55.8\% & 56.8\% & 100.0\% & 100.0\% & 100.0\% & 90.0\%\\ 
PDFBOX-265  &  org.apache.pdfbox.pdmodel.font.PDType3Font & 62.4\% & 70.2\% & 42.3\% & 52.0\% & \cellcolor{light-gray} \textcolor{black}{0.0\%} & \cellcolor{light-gray} \textcolor{black}{0.0\%}\\ 
PDFBOX-278  &  org.apache.pdfbox.pdfwriter.ContentStreamWriter & 96.8\% & 98.3\% & 96.7\% & 96.3\% & \cellcolor{light-gray} \textcolor{black}{0.0\%} & \cellcolor{light-gray} \textcolor{black}{0.0\%}\\ 
PDFBOX-285  &  org.apache.pdfbox.pdmodel.interactive.digitalsignature.PDSignature & 98.9\% & 99.7\% & 89.5\% & 95.5\% & \cellcolor{light-gray} \textcolor{black}{0.0\%} & \cellcolor{light-gray} \textcolor{black}{0.0\%}\\ 
PDFBOX-40  &  org.apache.pdfbox.pdmodel.font.PDCIDFontType2 & 57.2\% & 54.9\% & 45.1\% & 46.6\% & \cellcolor{light-gray} \textcolor{black}{0.0\%} & \cellcolor{light-gray} \textcolor{black}{0.0\%}\\ 
PDFBOX-62  &  org.apache.pdfbox.rendering.PageDrawer & 2.3\% & 6.8\% & 1.2\% & 4.2\% & \cellcolor{light-gray} \textcolor{black}{0.0\%} & \cellcolor{light-gray} \textcolor{black}{0.0\%}\\ 
PDFBOX-8  &  org.apache.pdfbox.pdmodel.font.FileSystemFontProvider & 45.2\% & 48.4\% & 34.2\% & 35.8\% & 41.9\% & 52.2\%\\ 
PDFBOX-83  &  org.apache.pdfbox.contentstream.operator.text.SetTextRenderingMode & 89.3\% & 85.7\% & 92.5\% & 100.0\% & 82.5\% & 87.5\%\\ 
PDFBOX-91  &  org.apache.pdfbox.pdmodel.documentinterchange.taggedpdf.PDArtifactMarkedContent & 91.6\% & 97.9\% & 71.2\% & 92.5\% & \cellcolor{light-gray} \textcolor{black}{0.0\%} & \cellcolor{light-gray} \textcolor{black}{0.0\%}\\ 
SPOON-105  &  spoon.support.compiler.jdt.PositionBuilder & 9.6\% & 5.5\% & 7.8\% & 3.9\% & \cellcolor{light-gray} \textcolor{black}{0.0\%} & \cellcolor{light-gray} \textcolor{black}{0.0\%}\\ 
SPOON-155  &  spoon.reflect.visitor.filter.AllMethodsSameSignatureFunction & 13.0\% & 12.7\% & \cellcolor{light-gray} \textcolor{black}{0.0\%} & 1.2\% & 0.7\% & 3.2\%\\ 
SPOON-16  &  spoon.reflect.path.CtElementPathBuilder & 15.9\% & 16.1\% & 8.0\% & 9.0\% & 10.3\% & 6.4\%\\ 
SPOON-169  &  spoon.reflect.visitor.ImportScannerImpl & 1.2\% & 10.6\% & 0.1\% & 4.7\% & \cellcolor{light-gray} \textcolor{black}{0.0\%} & 1.3\%\\ 
SPOON-20  &  spoon.support.reflect.reference.CtLocalVariableReferenceImpl & 30.0\% & 38.6\% & 14.0\% & 18.0\% & 3.3\% & 13.3\%\\ 
SPOON-211  &  spoon.reflect.path.impl.CtRolePathElement & 16.3\% & 18.3\% & 6.2\% & 10.3\% & 6.2\% & 11.2\%\\ 
SPOON-25  &  spoon.pattern.internal.ValueConvertorImpl & 3.0\% & 7.1\% & 1.2\% & 3.1\% & 0.7\% & 4.3\%\\ 
SPOON-253  &  spoon.pattern.internal.parameter.MapParameterInfo & 76.8\% & 73.9\% & 72.5\% & 73.8\% & \cellcolor{light-gray} \textcolor{black}{0.0\%} & \cellcolor{light-gray} \textcolor{black}{0.0\%}\\ 
SPOON-32  &  spoon.MavenLauncher & 27.0\% & 30.0\% & 11.2\% & 12.5\% & 6.0\% & 6.7\%\\ 
SPOON-65  &  spoon.support.DefaultCoreFactory & 10.7\% & 9.7\% & 5.9\% & 8.9\% & 0.1\% & \cellcolor{light-gray} \textcolor{black}{0.0\%}\\ 
\midrule 
Average &  &  55.9\% &  57.0\% &  50.8\% &  51.7\% &  32.6\% &  33.8\%\\ 
\bottomrule 
\end{tabular} 
}	
\end{table*}


\begin{table*}[t]
  \centering
  \caption{\label{table:results}Detailed results of \EVOSUITE on the SBST benchmark classes. Time is expressed in minutes.}
  %\scriptsize
\resizebox{0.9\textwidth}{!}{  
  \begin{tabular}{l rrrrrr} \toprule
Class &  JUnit Files &   Generation Time & Execution Time   & \% Line Coverage & \% Branch Coverage & \% Mutation Score \\  
\midrule
%com.googlecode.sqlsheet.stream.XlsSheetIterator & 217.65 & 0.00 & 1.00 & 0.0000 & 0.0000 & 0.0000 \\ 
com.googlecode.sqlsheet.stream.XlsxSheetIterator & 216.29 & 0.00 & 1.00 & 0.0000 & 0.0000 & 0.0000 \\ 
net.sourceforge.barbecue.Barcode & 195.05 & 0.00 & 1.00 & 0.0000 & 0.0000 & 0.0000 \\ 
net.sourceforge.barbecue.BlankModule & 193.12 & 0.26 & 1.00 & 1.0000 & 1.0000 & 0.1825 \\ 
net.sourceforge.barbecue.CompositeModule & 194.58 & 0.08 & 2.17 & 1.0000 & 1.0000 & 0.3527 \\ 
net.sourceforge.barbecue.Module & 192.90 & 0.03 & 3.67 & 0.8762 & 0.8611 & 0.2885 \\ 
net.sourceforge.barbecue.Modulo10 & 186.83 & 0.00 & 1.33 & 0.8667 & 1.0000 & 0.7971 \\ 
net.sourceforge.barbecue.SeparatorModule & 194.15 & 0.05 & 1.33 & 1.0000 & 1.0000 & 0.2012 \\ 
net.sourceforge.barbecue.env.DefaultEnvironment & 188.56 & 0.14 & 1.00 & 1.0000 & 1.0000 & 1.0000 \\ 
net.sourceforge.barbecue.env.EnvironmentFactory & 189.45 & 0.01 & 1.83 & 0.7593 & 0.6667 & 0.2436 \\ 
net.sourceforge.barbecue.env.HeadlessEnvironment & 186.99 & 0.00 & 1.00 & 1.0000 & 1.0000 & 1.0000 \\ 
net.sourceforge.barbecue.linear.LinearBarcode & 194.73 & 0.00 & 1.00 & 0.0000 & 0.0000 & 0.0000 \\ 
net.sourceforge.barbecue.linear.codabar.CodabarBarcode & 194.15 & 0.01 & 1.00 & 0.2738 & 0.1319 & 0.0000 \\ 
net.sourceforge.barbecue.linear.code128.Code128Barcode & 194.80 & 0.01 & 2.00 & 0.2548 & 0.0308 & 0.0000 \\ 
net.sourceforge.barbecue.linear.code128.ModuleFactory & 196.60 & 0.01 & 1.83 & 0.9944 & 0.9000 & 0.0008 \\ 
net.sourceforge.barbecue.linear.code39.Code39Barcode & 193.88 & 0.02 & 3.00 & 0.5109 & 0.5625 & 0.0000 \\ 
net.sourceforge.barbecue.linear.ean.UCCEAN128Barcode & 194.93 & 0.03 & 5.00 & 0.3591 & 0.1404 & 0.0000 \\ 
net.sourceforge.barbecue.linear.twoOfFive.Int2of5Barcode & 193.77 & 0.01 & 2.00 & 0.2667 & 0.5000 & 0.0000 \\ 
net.sourceforge.barbecue.linear.twoOfFive.Std2of5Barcode & 193.61 & 0.01 & 2.00 & 0.4136 & 0.4833 & 0.0139 \\ 
net.sourceforge.barbecue.output.GraphicsOutput & 192.90 & 0.15 & 3.33 & 0.8444 & 0.6333 & 0.1411 \\ 
org.apache.commons.lang3.ArrayUtils & 200.56 & 0.01 & 12.33 & 0.1097 & 0.0805 & 0.0000 \\ 
org.apache.commons.lang3.BooleanUtils & 12.55 & 0.00 & 0.00 & 0.0000 & 0.0000 & 0.0000 \\ 
org.apache.commons.lang3.CharRange & 187.10 & 0.00 & 13.67 & 0.9778 & 0.9400 & 0.4729 \\ 
org.apache.commons.lang3.math.Fraction & 195.11 & 0.01 & 35.67 & 0.9497 & 0.8861 & 0.1494 \\ 
org.apache.commons.lang3.math.NumberUtils & 190.04 & 0.01 & 8.83 & 0.1422 & 0.1183 & 0.0000 \\ 
org.apache.lucene.util.FixedBitSet & 192.46 & 0.03 & 41.50 & 0.9158 & 0.5173 & 0.0000 \\ 
org.apache.lucene.util.WeakIdentityMap & 186.98 & 0.01 & 4.83 & 0.9032 & 0.5000 & 0.0000 \\ 
org.joda.time.Chronology & 190.47 & 0.07 & 1.00 & 0.1053 & 1.0000 & 0.0000 \\ 
org.joda.time.DateTimeComparator & 191.00 & 0.06 & 9.33 & 0.9107 & 0.8225 & 0.0040 \\ 
org.joda.time.DateTimeFieldType & 189.80 & 0.07 & 10.83 & 1.0000 & 1.0000 & 0.0075 \\ 
org.joda.time.DateTimeUtils & 191.04 & 0.08 & 9.33 & 0.5653 & 0.4912 & 0.0040 \\ 
org.joda.time.DateTimeZone & 221.39 & 0.08 & 21.83 & 0.5161 & 0.4589 & 0.0052 \\ 
org.joda.time.Days & 191.94 & 0.17 & 18.67 & 0.8680 & 0.8632 & 0.0045 \\ 
org.joda.time.DurationField & 186.85 & 0.00 & 1.00 & 0.0000 & 0.0000 & 0.0028 \\ 
org.joda.time.DurationFieldType & 189.61 & 0.04 & 4.83 & 0.9944 & 1.0000 & 0.0021 \\ 
org.joda.time.Hours & 190.23 & 0.09 & 18.17 & 0.7764 & 0.7500 & 0.0040 \\ 
org.joda.time.IllegalFieldValueException & 187.27 & 0.01 & 11.50 & 0.4474 & 0.5625 & 0.0035 \\ 
org.joda.time.Minutes & 190.67 & 0.09 & 16.50 & 0.8213 & 0.8238 & 0.0037 \\ 
org.joda.time.Months & 190.17 & 0.08 & 20.00 & 0.8586 & 0.8485 & 0.0049 \\ 
org.joda.time.MutableDateTime & 195.96 & 0.33 & 56.50 & 0.3057 & 0.2027 & 0.0000 \\ 
org.joda.time.PeriodType & 192.25 & 0.05 & 13.00 & 0.4886 & 0.3494 & 0.0050 \\ 
org.joda.time.Seconds & 190.79 & 0.09 & 17.33 & 0.8478 & 0.8333 & 0.0042 \\ 
org.joda.time.Years & 190.04 & 0.08 & 15.17 & 0.8497 & 0.8571 & 0.0055 \\ 
org.joda.time.chrono.BuddhistChronology & 6.07 & 0.00 & 0.00 & 0.0000 & 0.0000 & 0.0000 \\ 
org.joda.time.chrono.GJChronology & 216.55 & 0.07 & 21.33 & 0.7836 & 0.5909 & 0.0205 \\ 
org.joda.time.chrono.GregorianChronology & 6.14 & 0.00 & 0.00 & 0.0000 & 0.0000 & 0.0000 \\ 
org.joda.time.chrono.ISOChronology & 189.47 & 0.04 & 5.33 & 0.8444 & 0.5833 & 0.0186 \\ 
org.joda.time.chrono.LenientChronology & 190.50 & 0.04 & 2.00 & 0.4967 & 0.1146 & 0.0165 \\ 
org.joda.time.convert.CalendarConverter & 191.64 & 0.05 & 3.33 & 0.5667 & 0.4500 & 0.4086 \\ 
org.joda.time.convert.ConverterManager & 189.47 & 0.02 & 12.33 & 0.6492 & 0.4624 & 0.1233 \\ 
org.joda.time.convert.ConverterSet & 190.54 & 0.05 & 9.00 & 0.6709 & 0.5979 & 0.4240 \\ 
org.joda.time.convert.DateConverter & 189.46 & 0.01 & 1.00 & 0.6667 & 1.0000 & 0.2353 \\ 
org.joda.time.convert.LongConverter & 189.43 & 0.02 & 1.33 & 0.9167 & 1.0000 & 0.4815 \\ 
org.joda.time.convert.NullConverter & 189.47 & 0.05 & 1.00 & 0.9091 & 1.0000 & 0.6439 \\ 
org.joda.time.convert.ReadableDurationConverter & 189.92 & 0.04 & 1.33 & 0.8889 & 0.8333 & 0.3125 \\ 
org.joda.time.convert.ReadableInstantConverter & 190.91 & 0.05 & 3.50 & 0.4537 & 0.2292 & 0.2262 \\ 
org.joda.time.convert.ReadableIntervalConverter & 190.57 & 0.06 & 4.33 & 0.6429 & 0.5278 & 0.5808 \\ 
org.joda.time.convert.ReadablePartialConverter & 190.73 & 0.05 & 1.83 & 0.5104 & 0.3750 & 0.3214 \\ 
org.joda.time.convert.ReadablePeriodConverter & 189.74 & 0.05 & 1.33 & 1.0000 & 1.0000 & 0.5000 \\ 
org.joda.time.convert.StringConverter & 191.44 & 0.07 & 16.17 & 0.5056 & 0.4571 & 0.0883 \\ 
org.joda.time.field.BaseDateTimeField & 194.84 & 0.07 & 17.50 & 0.8363 & 0.7986 & 0.0753 \\ 
org.joda.time.field.FieldUtils & 189.50 & 0.06 & 25.83 & 0.9262 & 0.9195 & 0.2500 \\ 
org.joda.time.field.MillisDurationField & 186.89 & 0.00 & 1.00 & 0.0000 & 0.0000 & 0.0394 \\ 
org.joda.time.field.OffsetDateTimeField & 188.86 & 0.05 & 3.17 & 0.4402 & 0.6250 & 0.0681 \\ 
org.joda.time.field.PreciseDateTimeField & 189.55 & 0.04 & 1.00 & 0.4683 & 0.3889 & 0.0124 \\ 
org.joda.time.field.PreciseDurationDateTimeField & 191.30 & 0.05 & 8.17 & 0.9383 & 0.8500 & 0.0771 \\ 
org.joda.time.field.PreciseDurationField & 186.85 & 0.00 & 1.00 & 0.0417 & 0.0000 & 0.0672 \\ 
org.joda.time.field.ScaledDurationField & 187.39 & 0.00 & 5.67 & 0.6579 & 0.5000 & 0.0717 \\ 
org.joda.time.field.UnsupportedDateTimeField & 193.01 & 0.07 & 38.83 & 0.7295 & 0.9306 & 0.0627 \\ 
org.joda.time.format.DateTimeFormat & 196.14 & 0.06 & 25.83 & 0.7200 & 0.5641 & 0.0142 \\ 
org.joda.time.format.DateTimeFormatter & 200.64 & 0.07 & 21.50 & 0.7480 & 0.6348 & 0.0205 \\ 
org.joda.time.format.DateTimeFormatterBuilder & 229.80 & 0.00 & 0.00 & 0.0000 & 0.0000 & 0.0000 \\ 
org.joda.time.format.ISODateTimeFormat & 191.23 & 0.04 & 18.33 & 0.7710 & 0.5076 & 0.0213 \\ 
org.joda.time.format.ISOPeriodFormat & 190.74 & 0.09 & 3.67 & 1.0000 & 1.0000 & 0.0341 \\ 
org.joda.time.format.PeriodFormat & 189.58 & 0.03 & 1.33 & 1.0000 & 1.0000 & 0.0293 \\ 
org.joda.time.format.PeriodFormatter & 191.23 & 0.05 & 8.33 & 0.9722 & 0.9318 & 0.0335 \\ 
org.joda.time.format.PeriodFormatterBuilder & 207.49 & 0.07 & 31.67 & 0.7805 & 0.6519 & 0.0393 \\ 

\midrule
%Average & & 6.19 & 0.03 & 55.36 & 47.19 & 41.02\\
\bottomrule
  \end{tabular}
}
\end{table*}



The results of \EVOSUITE on the benchmark classes are listed in
Table~\ref{table:results}. On average, \EVOSUITE achieved []\% line
coverage, []\% branch coverage\footnote{Using Cobertura's definition
  of branch coverage, which only counts conditional statements, not
  edges in the CFG.}, and []\% mutation score. 
 


\subsection{Issues Encountered}

Following recent experiments on
Defects4J~\cite{shamshiri2015automatically} that revealed general
issues in unit test generation, \EVOSUITE now includes assertions on
the source of exceptions, similar to commercial tools like Agitar
One~\cite{agitarone}. However, there were several instances in the
competition where these assertions lead to flaky tests. For example,
the following is an excerpt from a test for the Defects4J bug Lang-41,
generated by \EVOSUITE:

\begin{lstlisting}
@Test(timeout = 4000)
public void test19()  throws Throwable  {
 Class<Double> class0 = Double.class;
 String string0 = ClassUtils.getPackageName(class0);
 try { 
   ClassUtils.getClass(string0);
   fail("Expecting exception: ClassNotFoundException"); 
 } catch(ClassNotFoundException e) {
   assertThrownBy("java.net.URLClassLoader", e);
 }
}
\end{lstlisting}

While compiling and executing this test with JUnit works without
problems, the mutation analysis step of the competition used Ant to
run the tests; Ant uses a complex setup of classloaders that
eventually leads to the \texttt{assertThrownBy} in the above example
to fail, as the source of the exception is a different one.


%------------------------------------------------------------------------- 
\section{Conclusions}

With an overall score of [], \EVOSUITE achieved the []
 score of all tools in the competition. 


To learn more about \EVOSUITE, visit our Web site:
\begin{center}
%\url{http://evosuite.org/}
\texttt{http://www.evosuite.org}
\end{center}


%------------------------------------------------------------------------- 

%\noindent
\textbf{Acknowledgments.} 

Many thanks to all the contributors to \EVOSUITE.
This project has been funded by 
%the EPSRC project ``EXOGEN'' (EP/K030353/1), a Google Focused Research Award on ``Test
%Amplification'', and by 
the National Research Fund, Luxembourg (FNR/P10/03).


%------------------------------------------------------------------------- 
%\def\IEEEbibitemsep{5pt plus 1pt}
\def\IEEEbibitemsep{6pt}

\bibliographystyle{IEEEtranS}
\bibliography{papers}

\end{document}

