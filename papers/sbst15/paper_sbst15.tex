%--------1---------2---------3---------4---------5---------6---------7
%
% Competition paper
% SBST 2013
% Page limit: 4 pages
%

%\documentclass[10pt, conference, compsocconf]{IEEEtran}
\documentclass[10pt,conference,compsocconf]{IEEEtran}

%\documentclass[times, 10pt,twocolumn]{article} 
%\usepackage{latex8}
%\usepackage{times}
\usepackage{amsfonts}
\usepackage[latin1]{inputenc}
\usepackage[english]{babel}
\usepackage{listings}
\usepackage{algorithmic}
\usepackage{float}
\usepackage{cite}
\usepackage{graphicx}
\usepackage{booktabs}
\usepackage{subfigure}
\usepackage{hyperref}
\usepackage{color}
\usepackage[usenames,dvipsnames,table]{xcolor}
\usepackage{soul}
\usepackage{xspace}
\usepackage{boxedminipage}
\usepackage{alltt}
\usepackage{multirow}
\usepackage{paralist}
\usepackage{amsmath}



\floatstyle{ruled}
\newfloat{algorithm}{tbp}{loa}
\floatname{algorithm}{Algorithm}

\newtheorem{definition}{Definition}


 %krams hinter fontadjust ist neu
  \definecolor{lightgrey}{rgb}{0.90,0.90,0.90}
\lstset{escapeinside={(*}{*)}}
  \lstloadlanguages{java}
 \lstdefinelanguage{pseudocode}
  {morekeywords={if, else, initialize, return, for, each, in, global, new}
   }
  \lstset{
    tabsize=2,
    mathescape=true,
    escapeinside={(*}{*)},
    captionpos=t,
    framerule=0pt,
    backgroundcolor=\color{lightgrey},
    basicstyle=\footnotesize\ttfamily,
    keywordstyle=\footnotesize\bfseries,
    numbers=none,
    numberstyle=\tiny,
    numbersep=1pt,
    fontadjust,
    breaklines=true,
    breakatwhitespace=false
  }    
      

\hypersetup{
colorlinks=true,
urlcolor=rltblue,
linkcolor=rltred,
citecolor=rltgreen,
bookmarksnumbered=true,
pdftitle={EvoSuite at the SBST 2013 Tool Competition},
pdfauthor={Gordon Fraser and Andrea Arcuri},
pdfsubject={Test case generation},
pdfkeywords={Test case generation, unit testing, test
  oracles, assertions, search based testing}
}

\definecolor{rltred}{rgb}{0.5,0,0}
\definecolor{rltgreen}{rgb}{0,0.5,0}
\definecolor{rltblue}{rgb}{0,0,0.5}
\definecolor{ScarletRed}{rgb}{0.80,0.00,0.00}



% in draft mode we put \remarks into the margins and do other stuff
% set to \draftfalse for 
\newif\ifdraft
\draftfalse

\ifdraft
	\marginparwidth=1.3cm
	\marginparsep=5pt
	\newcommand\remark[1]{%
		\mymarginpar{\raggedright\hbadness=10000\tiny\it #1\par}}
\else
	\newcommand\remark[1]	{}
\fi

\ifdraft
	\overfullrule3pt
\fi    

% We use \FIXME for located problems (``defect'')
\newcommand{\FIXME}[1]{\remark{FIXME: #1}}
\newcommand\parremark[1]	{\par\textbf{REMARK:} #1\par}

\newcommand{\gordon}[1]{\textcolor{blue}{\sf\small\textbf{Gordon:} #1}}
\newcommand{\andrea}[1]{\textcolor{ScarletRed}{\sf\small\textbf{Andrea:} #1}}

% \mathid is used to denote identifiers and slots in formulas
\newcommand{\mathid}[1]{\text{\rmfamily\textit{#1}}}

% But usually, we shall use \|name| instead.
\def\|#1|{\mathid{#1}}

% \codeid is used to denote computer code identifiers
\newcommand{\codeid}[1]{\texttt{#1}}

% But usually, we shall use \<name> instead.
\def\<#1>{\codeid{#1}}

% Our results
\newenvironment{result}%
{\smallskip
\noindent
\let\emph=\textbf
\begin{boxedminipage}{\columnwidth}\begin{center}\em}%
{\end{center}\end{boxedminipage}%
\smallskip
}

\newcommand{\JodaTime}{Joda-Time\xspace}  % That's how they write themselves -- AZ

\newcommand{\EVOSUITE}{{\sc EvoSuite}\xspace}
\newcommand{\MUTEST}{{\sc $\mu$Test}\xspace}
\newcommand{\CS}{{\sc SF100}\xspace}

% TODO marker
\newcommand{\TODO}[1]{\sethlcolor{yellow}\textbf{\textcolor{ScarletRed}{\hl{TODO: #1}}}\xspace}


\DeclareMathSymbol{,}{\mathpunct}{letters}{"3B}
\DeclareMathSymbol{,}{\mathord}{letters}{"3B}
\DeclareMathSymbol{\decimal}{\mathord}{letters}{"3A}
%%%"

%\documentstyle[times,art10,twocolumn,latex8]{article}

%------------------------------------------------------------------------- 
% take the % away on next line to produce the final camera-ready version 
%\pagestyle{empty}

%------------------------------------------------------------------------- 
\begin{document}

%\title{Unit Testing Tool competition: Results for EvoSuite}
\title{EvoSuite at the SBST 2015 Tool Competition}

\author{
\IEEEauthorblockN{Gordon Fraser}
\IEEEauthorblockA{University of Sheffield\\
Sheffield, UK}\\
%gordon.fraser@sheffield.ac.uk}\\
\and
\IEEEauthorblockN{Andrea Arcuri}
\IEEEauthorblockA{Scienta, Norway\\
and University of Luxembourg}\\
%arcuri@simula.no}
%\and
%\IEEEauthorblockN{Jeremias R\"{o}{\ss}ler}
%\IEEEauthorblockA{Saarland University -- Computer Science\\
%Saarbr\"ucken, Germany\\
%roessler@cs.uni-saarland.de}
}

\maketitle
%\thispagestyle{empty}

\begin{abstract}
  \EVOSUITE is a mature research prototype that automatically generates unit tests for Java code. 
  This paper summarizes the results and experiences in participating at the third unit testing competition held at SBST 2015,
  where \EVOSUITE ranked second with a $190.6$ score.
\end{abstract}

\begin{IEEEkeywords}
  test case generation; search-based testing; testing classes;
  search-based software engineering
\end{IEEEkeywords}


%------------------------------------------------------------------------- 
\section{Introduction}

This paper describes the results of applying the \EVOSUITE test
generation tool~\cite{FrA11c} to the benchmark used in the tool
competition at the International Workshop on Search-Based Software
Testing (SBST) 2015.  Details on the competition and the benchmark can
be found in~\cite{sbst2015}. In this competition, \EVOSUITE ranked
second with a score of $190.6$.

%------------------------------------------------------------------------- 
\section{About \EVOSUITE}


\begin{table}[!h]
\renewcommand{\arraystretch}{1.3}
\caption{Classification of the \EVOSUITE unit test generation tool.}\label{tool-description}
\begin{tabular}{|l|p{5cm}|}
  \hline
  \multicolumn{2}{|l|}{Prerequisites} \\
  \hline
  Static or dynamic &  Dynamic testing at the Java class level\\
  Software Type &  Java classes\\
  Lifecycle phase&  Unit testing for Java programs\\
  Environment&  All Java development environments \\
  Knowledge required & JUnit unit testing for Java\\
  Experience required &  Basic unit testing knowledge\\
 \hline
  \multicolumn{2}{|l|}{Input and Output of the tool} \\
  \hline
 Input & Bytecode of the target class and dependencies \\
\hline
Output&  JUnit test cases (version 3 or 4)\\
 
  \hline
  \multicolumn{2}{|l|}{Operation} \\
  \hline
  Interaction &  Through the command line\\
  User guidance &  manual verification of assertions for functional faults\\
  Source of information &  http://www.evosuite.org \\
  Maturity&  Mature research prototype, under development\\
  Technology behind the tool & Search-based testing / whole test suite generation\\
\hline
  \multicolumn{2}{|l|}{Obtaining the tool and information} \\
  \hline
License & GNU General Public License V3\\
Cost & Open source\\
Support & None \\
\hline
\hline
  \multicolumn{2}{|l|}{Does there exist empirical evidence about} \\
  \hline
  Effectiveness and Scalability & See~\cite{GoA_TSE12,FrA12b}. \\
%Completeness & \\
%Effectiveness & \\
%Efficiency & \\
%Defect types & \\
%Scalability & \\
%Comprehensibility & \\
%Learnability & \\
%Subjective satisfaction & \\
%Other & \\
\hline
\end{tabular}\vspace{-1em}
\end{table}



%------------------------------------------------------------------------- 
\section{Benchmark results}

\begin{table*}[t]
  \centering
  \caption{\label{table:results}Detailed results of \EVOSUITE on the SBST benchmark classes.}
  %\scriptsize
\resizebox{0.9\textwidth}{!}{  
  \begin{tabular}{l rrrrrr} \toprule
Class &  JUnit Files &   Generation Time & Execution Time   & \% Line Coverage & \% Branch Coverage & \% Mutation Score \\  
% Class & Generation Time & Execution Time  & Tests & Line Coverage & Branch Coverage & Mutation Score \\ 
%Class & \multicolumn{2}{c}{ Time} & Tests & Line Coverage & Branch
%Coverage & Mutation Score \\ 
%& Generation & Execution & & & & \\
\midrule
com.googlecode.sqlsheet.stream.XlsSheetIterator & 217.65 & 0.00 & 1.00 & 0.0000 & 0.0000 & 0.0000 \\ 
com.googlecode.sqlsheet.stream.XlsxSheetIterator & 216.29 & 0.00 & 1.00 & 0.0000 & 0.0000 & 0.0000 \\ 
net.sourceforge.barbecue.Barcode & 195.05 & 0.00 & 1.00 & 0.0000 & 0.0000 & 0.0000 \\ 
net.sourceforge.barbecue.BlankModule & 193.12 & 0.26 & 1.00 & 1.0000 & 1.0000 & 0.1825 \\ 
net.sourceforge.barbecue.CompositeModule & 194.58 & 0.08 & 2.17 & 1.0000 & 1.0000 & 0.3527 \\ 
net.sourceforge.barbecue.Module & 192.90 & 0.03 & 3.67 & 0.8762 & 0.8611 & 0.2885 \\ 
net.sourceforge.barbecue.Modulo10 & 186.83 & 0.00 & 1.33 & 0.8667 & 1.0000 & 0.7971 \\ 
net.sourceforge.barbecue.SeparatorModule & 194.15 & 0.05 & 1.33 & 1.0000 & 1.0000 & 0.2012 \\ 
net.sourceforge.barbecue.env.DefaultEnvironment & 188.56 & 0.14 & 1.00 & 1.0000 & 1.0000 & 1.0000 \\ 
net.sourceforge.barbecue.env.EnvironmentFactory & 189.45 & 0.01 & 1.83 & 0.7593 & 0.6667 & 0.2436 \\ 
net.sourceforge.barbecue.env.HeadlessEnvironment & 186.99 & 0.00 & 1.00 & 1.0000 & 1.0000 & 1.0000 \\ 
net.sourceforge.barbecue.linear.LinearBarcode & 194.73 & 0.00 & 1.00 & 0.0000 & 0.0000 & 0.0000 \\ 
net.sourceforge.barbecue.linear.codabar.CodabarBarcode & 194.15 & 0.01 & 1.00 & 0.2738 & 0.1319 & 0.0000 \\ 
net.sourceforge.barbecue.linear.code128.Code128Barcode & 194.80 & 0.01 & 2.00 & 0.2548 & 0.0308 & 0.0000 \\ 
net.sourceforge.barbecue.linear.code128.ModuleFactory & 196.60 & 0.01 & 1.83 & 0.9944 & 0.9000 & 0.0008 \\ 
net.sourceforge.barbecue.linear.code39.Code39Barcode & 193.88 & 0.02 & 3.00 & 0.5109 & 0.5625 & 0.0000 \\ 
net.sourceforge.barbecue.linear.ean.UCCEAN128Barcode & 194.93 & 0.03 & 5.00 & 0.3591 & 0.1404 & 0.0000 \\ 
net.sourceforge.barbecue.linear.twoOfFive.Int2of5Barcode & 193.77 & 0.01 & 2.00 & 0.2667 & 0.5000 & 0.0000 \\ 
net.sourceforge.barbecue.linear.twoOfFive.Std2of5Barcode & 193.61 & 0.01 & 2.00 & 0.4136 & 0.4833 & 0.0139 \\ 
net.sourceforge.barbecue.output.GraphicsOutput & 192.90 & 0.15 & 3.33 & 0.8444 & 0.6333 & 0.1411 \\ 
org.apache.commons.lang3.ArrayUtils & 200.56 & 0.01 & 12.33 & 0.1097 & 0.0805 & 0.0000 \\ 
org.apache.commons.lang3.BooleanUtils & 12.55 & 0.00 & 0.00 & 0.0000 & 0.0000 & 0.0000 \\ 
org.apache.commons.lang3.CharRange & 187.10 & 0.00 & 13.67 & 0.9778 & 0.9400 & 0.4729 \\ 
org.apache.commons.lang3.math.Fraction & 195.11 & 0.01 & 35.67 & 0.9497 & 0.8861 & 0.1494 \\ 
org.apache.commons.lang3.math.NumberUtils & 190.04 & 0.01 & 8.83 & 0.1422 & 0.1183 & 0.0000 \\ 
org.apache.lucene.util.FixedBitSet & 192.46 & 0.03 & 41.50 & 0.9158 & 0.5173 & 0.0000 \\ 
org.apache.lucene.util.WeakIdentityMap & 186.98 & 0.01 & 4.83 & 0.9032 & 0.5000 & 0.0000 \\ 
org.joda.time.Chronology & 190.47 & 0.07 & 1.00 & 0.1053 & 1.0000 & 0.0000 \\ 
org.joda.time.DateTimeComparator & 191.00 & 0.06 & 9.33 & 0.9107 & 0.8225 & 0.0040 \\ 
org.joda.time.DateTimeFieldType & 189.80 & 0.07 & 10.83 & 1.0000 & 1.0000 & 0.0075 \\ 
org.joda.time.DateTimeUtils & 191.04 & 0.08 & 9.33 & 0.5653 & 0.4912 & 0.0040 \\ 
org.joda.time.DateTimeZone & 221.39 & 0.08 & 21.83 & 0.5161 & 0.4589 & 0.0052 \\ 
org.joda.time.Days & 191.94 & 0.17 & 18.67 & 0.8680 & 0.8632 & 0.0045 \\ 
org.joda.time.DurationField & 186.85 & 0.00 & 1.00 & 0.0000 & 0.0000 & 0.0028 \\ 
org.joda.time.DurationFieldType & 189.61 & 0.04 & 4.83 & 0.9944 & 1.0000 & 0.0021 \\ 
org.joda.time.Hours & 190.23 & 0.09 & 18.17 & 0.7764 & 0.7500 & 0.0040 \\ 
org.joda.time.IllegalFieldValueException & 187.27 & 0.01 & 11.50 & 0.4474 & 0.5625 & 0.0035 \\ 
org.joda.time.Minutes & 190.67 & 0.09 & 16.50 & 0.8213 & 0.8238 & 0.0037 \\ 
org.joda.time.Months & 190.17 & 0.08 & 20.00 & 0.8586 & 0.8485 & 0.0049 \\ 
org.joda.time.MutableDateTime & 195.96 & 0.33 & 56.50 & 0.3057 & 0.2027 & 0.0000 \\ 
org.joda.time.PeriodType & 192.25 & 0.05 & 13.00 & 0.4886 & 0.3494 & 0.0050 \\ 
org.joda.time.Seconds & 190.79 & 0.09 & 17.33 & 0.8478 & 0.8333 & 0.0042 \\ 
org.joda.time.Years & 190.04 & 0.08 & 15.17 & 0.8497 & 0.8571 & 0.0055 \\ 
org.joda.time.chrono.BuddhistChronology & 6.07 & 0.00 & 0.00 & 0.0000 & 0.0000 & 0.0000 \\ 
org.joda.time.chrono.GJChronology & 216.55 & 0.07 & 21.33 & 0.7836 & 0.5909 & 0.0205 \\ 
org.joda.time.chrono.GregorianChronology & 6.14 & 0.00 & 0.00 & 0.0000 & 0.0000 & 0.0000 \\ 
org.joda.time.chrono.ISOChronology & 189.47 & 0.04 & 5.33 & 0.8444 & 0.5833 & 0.0186 \\ 
org.joda.time.chrono.LenientChronology & 190.50 & 0.04 & 2.00 & 0.4967 & 0.1146 & 0.0165 \\ 
org.joda.time.convert.CalendarConverter & 191.64 & 0.05 & 3.33 & 0.5667 & 0.4500 & 0.4086 \\ 
org.joda.time.convert.ConverterManager & 189.47 & 0.02 & 12.33 & 0.6492 & 0.4624 & 0.1233 \\ 
org.joda.time.convert.ConverterSet & 190.54 & 0.05 & 9.00 & 0.6709 & 0.5979 & 0.4240 \\ 
org.joda.time.convert.DateConverter & 189.46 & 0.01 & 1.00 & 0.6667 & 1.0000 & 0.2353 \\ 
org.joda.time.convert.LongConverter & 189.43 & 0.02 & 1.33 & 0.9167 & 1.0000 & 0.4815 \\ 
org.joda.time.convert.NullConverter & 189.47 & 0.05 & 1.00 & 0.9091 & 1.0000 & 0.6439 \\ 
org.joda.time.convert.ReadableDurationConverter & 189.92 & 0.04 & 1.33 & 0.8889 & 0.8333 & 0.3125 \\ 
org.joda.time.convert.ReadableInstantConverter & 190.91 & 0.05 & 3.50 & 0.4537 & 0.2292 & 0.2262 \\ 
org.joda.time.convert.ReadableIntervalConverter & 190.57 & 0.06 & 4.33 & 0.6429 & 0.5278 & 0.5808 \\ 
org.joda.time.convert.ReadablePartialConverter & 190.73 & 0.05 & 1.83 & 0.5104 & 0.3750 & 0.3214 \\ 
org.joda.time.convert.ReadablePeriodConverter & 189.74 & 0.05 & 1.33 & 1.0000 & 1.0000 & 0.5000 \\ 
org.joda.time.convert.StringConverter & 191.44 & 0.07 & 16.17 & 0.5056 & 0.4571 & 0.0883 \\ 
org.joda.time.field.BaseDateTimeField & 194.84 & 0.07 & 17.50 & 0.8363 & 0.7986 & 0.0753 \\ 
org.joda.time.field.FieldUtils & 189.50 & 0.06 & 25.83 & 0.9262 & 0.9195 & 0.2500 \\ 
org.joda.time.field.MillisDurationField & 186.89 & 0.00 & 1.00 & 0.0000 & 0.0000 & 0.0394 \\ 
org.joda.time.field.OffsetDateTimeField & 188.86 & 0.05 & 3.17 & 0.4402 & 0.6250 & 0.0681 \\ 
org.joda.time.field.PreciseDateTimeField & 189.55 & 0.04 & 1.00 & 0.4683 & 0.3889 & 0.0124 \\ 
org.joda.time.field.PreciseDurationDateTimeField & 191.30 & 0.05 & 8.17 & 0.9383 & 0.8500 & 0.0771 \\ 
org.joda.time.field.PreciseDurationField & 186.85 & 0.00 & 1.00 & 0.0417 & 0.0000 & 0.0672 \\ 
org.joda.time.field.ScaledDurationField & 187.39 & 0.00 & 5.67 & 0.6579 & 0.5000 & 0.0717 \\ 
org.joda.time.field.UnsupportedDateTimeField & 193.01 & 0.07 & 38.83 & 0.7295 & 0.9306 & 0.0627 \\ 
org.joda.time.format.DateTimeFormat & 196.14 & 0.06 & 25.83 & 0.7200 & 0.5641 & 0.0142 \\ 
org.joda.time.format.DateTimeFormatter & 200.64 & 0.07 & 21.50 & 0.7480 & 0.6348 & 0.0205 \\ 
org.joda.time.format.DateTimeFormatterBuilder & 229.80 & 0.00 & 0.00 & 0.0000 & 0.0000 & 0.0000 \\ 
org.joda.time.format.ISODateTimeFormat & 191.23 & 0.04 & 18.33 & 0.7710 & 0.5076 & 0.0213 \\ 
org.joda.time.format.ISOPeriodFormat & 190.74 & 0.09 & 3.67 & 1.0000 & 1.0000 & 0.0341 \\ 
org.joda.time.format.PeriodFormat & 189.58 & 0.03 & 1.33 & 1.0000 & 1.0000 & 0.0293 \\ 
org.joda.time.format.PeriodFormatter & 191.23 & 0.05 & 8.33 & 0.9722 & 0.9318 & 0.0335 \\ 
org.joda.time.format.PeriodFormatterBuilder & 207.49 & 0.07 & 31.67 & 0.7805 & 0.6519 & 0.0393 \\ 

\midrule
Average & & 371.58 & 2.10 & 55.36 & 47.19 & 41.02\\
\bottomrule
  \end{tabular}
}
\end{table*}

%average per class generation time:                 371.581857 seconds
%average per class execution time:                  2.103966 seconds
%average per class instruction coverage:            54.495767 %
%average per class line coverage:                   55.368111 %
%average per class branch coverage:                 47.192896 %
%average per class method coverage:                 59.628209 %
%average per class mutation coverage:               41.024248 %

The results of \EVOSUITE on the benchmark classes are listed in
Table~\ref{table:results}. On average, \EVOSUITE achieved TODO line
coverage,TODO branch coverage\footnote{Using Cobertura's definition
  of branch coverage, which only counts conditional statements, not
  edges in the CFG.}, and TODO mutation score. On average, \EVOSUITE
produced TODO tests per class, and it took an average of TODO per class to
do so (with \EVOSUITE configured to 3 minutes per class).



\subsection{Classes with No Coverage}

Out of 63 CUTs, \EVOSUITE did not manage to obtain any coverage for 18 classes, i.e. 28\% 
of the total.
For 14 of them (the 7 in the twitter4j and the 7 in the hibernate projects), this is due
to a mismatch in libraries on the classpath. 
In particular, \EVOSUITE does bytecode instrumentation using the ASM library.
However, the projects of the CUTs had their own (older) version of ASM.
This leads to errors like: ``java.lang.IncompatibleClassChangeError: 
class org.objectweb.asm.tree.ClassNode has interface org.objectweb.asm.ClassVisitor as super class''.
Note: \EVOSUITE can handle its own dependencies as CUTs, i.e. by using customized classloaders, 
but only during the search phase.
In the generated JUnit files, those have dependencies to ASM, as we do bytecode instrumentation
even there (e.g., to support environment testing based on mock objects [TODO]).
So, \EVOSUITE does generate tests for those 14 classes, but then all those tests fails due to
that thrown exception.
An easy solution would be to ship \EVOSUITE with its own ASM version using a 
different package name (e.g., by using the JarJar tool\footnote{https://code.google.com/p/jarjar/}).

For the other 4 classes, \EVOSUITE failed to generate any test. These are
\<CharMatcher>, \<CycleHandler>, \<WikipediaInfo> and \<Response>.

TODO

There are also cases of low coverage (e.g., less than 30\%).
For example, if look at line coverage, then 
we have \<Page> with 1.49\%, \<AbstractInstance> with 13.28\%, \<CategoryDescendantsIterator> with 8.24\%
and \<PageQueryIterable> with 18.57\%.

TODO

%------------------------------------------------------------------------- 
\section{Conclusions}


To learn more about \EVOSUITE, visit our Web site:
\begin{center}
%\url{http://evosuite.org/}
\texttt{http://www.evosuite.org}
\end{center}


%------------------------------------------------------------------------- 

%\noindent
%\textbf{Acknowledgments.} 



%------------------------------------------------------------------------- 
%\def\IEEEbibitemsep{5pt plus 1pt}
\def\IEEEbibitemsep{6pt}

\bibliographystyle{IEEEtranS}
\bibliography{papers}

\end{document}

