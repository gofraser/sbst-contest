\documentclass[runningheads,a4paper]{llncs}

\usepackage{mathtools}
\usepackage{rotating}
\usepackage{longtable}
\usepackage{amssymb}
\setcounter{tocdepth}{3}
\usepackage{graphicx}
\usepackage[utf8]{inputenc}
\usepackage{url}
\usepackage{amsmath}
\usepackage{algorithm2e}
\usepackage{xspace}
\usepackage{cite}
\usepackage{microtype}
\usepackage{booktabs}
\usepackage{color}
\usepackage{colortbl}
\usepackage{tabularx}
\usepackage[usenames,dvipsnames,table]{xcolor}
\usepackage{ctable}

\usepackage[pdftex]{hyperref}
\hypersetup{colorlinks,% 
citecolor=black,% 
filecolor=black,% 
linkcolor=black,% 
urlcolor=black 
}

\setupctable{botcap, captionskip = 0.5ex}


\newcommand{\keywords}[1]{\par\addvspace\baselineskip
\noindent\keywordname\enspace\ignorespaces#1}

\newcommand{\EVOSUITE}{{\sc EvoSuite}\xspace}
\newcommand{\MUTEST}{{\sc $\mu$Test}\xspace}
\newcommand{\CS}{{\sc SF100}\xspace}


\begin{document}

\mainmatter
\title{EvoSuite at the Second \\ Unit Testing Tool Competition}
\author{Gordon Fraser\inst{1} \and Andrea Arcuri\inst{2}}
%
\institute{University of Sheffield\\
Dep. of Computer Science, Sheffield, UK\\
\email{gordon.fraser@sheffield.ac.uk},\\
\and
Simula Research Laboratory\\
P.O. Box 134, 1325 Lysaker, Norway\\
\email{arcuri@simula.no}
}


\maketitle

\begin{abstract}
  \EVOSUITE is a mature research prototype implementing a search-based
  approach to unit test generation for Java classes. It has been
  successfully run on a variety of different Java projects in the
  past, and after winning the first instance of the unit testing tool
  competition at SBST'13, it has also taken part in the second
  run. This paper reports on the obstacles and challenges encountered
  during the latter competition.

\keywords{automated unit testing, search-based testing, competition}
\end{abstract}


\section{Introduction}

The \EVOSUITE test generation tool is a mature research prototype that
automatically generates unit test suites for given Java classes. The
first experiments with \EVOSUITE were reported in~\cite{FrA11b}, and
it has since been applied to a range of different projects and
domains~\cite{GoA_TSE12,FrA12b}, leading to various improvements over
time. 
In the first unit test competition organised at the SBST'13 workshop~\cite{sbst2013},
\EVOSUITE obtained the highest score among the participating tools~\cite{evosuiteAtSbst2013}.

Besides the obvious research challenge of achieving high coverage, the
challenges in building a tool like \EVOSUITE often lie in practical
issues imposed by the Java language, and the nature of real code. For
example, often the structure of code is trivial in terms of the
complexity of the branching conditions, yet difficult for a testing
tool as the code has complex environmental dependencies, such as files
and databases~\cite{FrA13a}. These findings are once more confirmed by
the results of the second unit testing tool competition.
%, following that was run
%after the first one~\cite{evosuiteAtSbst2013}.  
In this paper, we analyze the major obstacles \EVOSUITE encountered in
this second competition by focusing on classes where the coverage is
particularly low.


\section{About \EVOSUITE}

\EVOSUITE is a tool that automatically produces unit test suites for
Java classes with the aim to maximize code coverage. As input it
requires only the bytecode of the class under test as well as its
dependencies. This is provided automatically when using the
Eclipse-frontend to interact with \EVOSUITE, and on the command-line
it amounts to setting a classpath and specifying the target class name
as a command-line parameter.

\begin{table}[t] %[!h]
\centering
    \renewcommand{\arraystretch}{1.2}

    \begin{tabular}{|l|p{5cm}|}
      \hline
      \multicolumn{2}{|l|}{{\bf Prerequisites}} \\
      \hline
  Static or dynamic &  Dynamic testing at the Java class level\\
  Software Type &  Java classes\\
  Lifecycle phase&  Unit testing for Java programs\\
  Environment&  All Java development environments \\
  Knowledge required & JUnit unit testing for Java\\
  Experience required &  Basic unit testing knowledge\\
     \hline
      \multicolumn{2}{|l|}{{\bf Input and Output of the tool}} \\
      \hline
 Input & Bytecode of the target class and dependencies \\
    \hline
Output&  JUnit test cases (version 3 or 4)\\
     
      \hline
      \multicolumn{2}{|l|}{{\bf Operation}} \\
      \hline
  Interaction &  Through the command line\\
  User guidance &  manual verification of assertions for functional faults\\
  Source of information &  http://www.evosuite.org \\
  Maturity&  Mature research prototype, under development\\
  Technology behind the tool & Search-based testing / whole test suite generation\\
    \hline
      \multicolumn{2}{|l|}{{\bf Obtaining the tool and information}} \\
      \hline
License & GNU General Public License V3\\
Cost & Open source\\
Support & None \\
    \hline
    \hline
      \multicolumn{2}{|l|}{{\bf Empirical evidence about the tool}} \\
      \hline
  Effectiveness and Scalability & See~\cite{GoA_TSE12,FrA12b}. \\
    \hline
    \end{tabular}
%
\vspace{0.4cm}
	\caption{Description of \EVOSUITE}\label{tool-description}
\end{table}


\EVOSUITE applies a search-based approach, where a genetic algorithm
optimizes whole test suites towards a chosen target criterion (e.g.,
branch coverage by default). The advantage of this approach is that it
is not negatively affected by infeasible goals and its performance
does not depend on the order in which testing goals are considered, as
has been demonstrated in the past~\cite{GoA_TSE12}. To increase the
performance further, \EVOSUITE integrates dynamic symbolic execution
(DSE) in a hybrid approach~\cite{evoISSRE113}, where primitive values
(e.g., numbers or strings) are optimized using a constraint
solver. However, due to the experimental nature of this feature it was
not activated during the competition.

The search-based optimization in \EVOSUITE results in a set of
sequences of method calls that maximizes coverage, yet these sequences
are not yet usable as unit test cases --- at least not for human
consumption. In order to make it easier to understand and use the test
cases produced, \EVOSUITE applies a range of post-processing steps to
produce sets of small and concise unit tests. The final step of this
post-processing consists of adding test assertions, i.e., statements
that check the outcome of the test. As \EVOSUITE only requires the
bytecode as input and such bytecode often does not include formal specifications, 
the assertions
produced will reflect observed behavior, rather than the intended
behavior. Consequently, the test cases are intended either as
regression tests, or serve as starting point for a human tester, who
can manually revise the assertions in order to determine
failures~\cite{10.1109/TSE.2011.93}. However, \EVOSUITE is also able
to automatically detect certain classes of bugs which are independent
of a specification; for example, undeclared exceptions or violations
of assertions in the code~\cite{emse13_oracle}.

Table \ref{tool-description} describes \EVOSUITE in terms of the
template used in the unit testing tool competition.



\section{Configuration for the Competition Entry}

Even though several new features have been developed for \EVOSUITE
(e.g., DSE integration~\cite{evoISSRE113}) since the last competition,
we did not include any for the competition entry, as the risk of
reducing the score due to immature code would be too high. One notable
new feature we did include because it is now enabled by default is
support for Java Generics~\cite{evoGenerics13}. Besides this, the
version of \EVOSUITE used in the competition is largely the same as in
the first round in terms of features, yet has seen many revisions to
fix individual problems or bugs.

We used the same configuration as for the first unit testing
competition without any changes. This means that \EVOSUITE was
configured to optimize for weak mutation testing, with three minutes
time for the search per class. Minimization was deactivated to reduce
the test generation time, and \EVOSUITE was configured to include all
possible assertions in the tests (rather than the default of
minimizing the assertions using mutation
analysis~\cite{10.1109/TSE.2011.93}). For all other parameters,
\EVOSUITE was configured to its default parameter
settings~\cite{arcuri2013parameter}.


\section{Results and Problems}

%\resizebox{\textwidth}{!}{%
{\tiny
\ctable[nostar, pos = t!, doinside=\rowcolors{3}{white}{gray!25},
  label = table:results, caption = {\normalfont \EVOSUITE results on the benchmark classes}
]{lllrr@{\hfill}rr@{\hfill}rrr}{}{ %
%\scalebox{0.5}{
%\begin{table}[h!]
%\centering
%\begin{tabular}{lllrrrrr}
  \FL
{\bf No.} & 
{\bf Class} & 
{\bf Project} & 
{\bf LOC} & 
{\bf NBD} & 
\multicolumn{2}{c}{\bf Time (min)}
&
\multicolumn{3}{c}{\bf Coverage}
 \NN
& & & & & 
{\bf Gen} & 
{\bf Exec} & 
{\bf Instr.} & 
{\bf Branch } & 
{\bf Mutation}
\ML

%Cyclo  &  LOC  &  TLOC & TASS
1 &SearchException & Hibernate & 15 & 1 &                                                3.23    &   0.00    &   0.00      &      0.00       &     0.00  \NN                      
2 &Version & Hibernate & 12 & 1 &                                                        3.74    &   0.01    &   60.00     &      0.00       &     0.00  \NN
3 &BackendFactory & Hibernate & 72 & 2 &                                                 4.53    &   0.02    &   27.71     &      6.25       &     40.95 \NN
4 &FlushLuceneWork & Hibernate & 26 & 2 &                                                3.22    &   0.03    &   89.74     &      100.00     &     75.00 \NN
5 &OptimizeLuceneWork & Hibernate & 26 & 2 &                                             3.22    &   0.03    &   89.74     &      100.00     &     75.00 \NN
6 &LoggerFactory & Hibernate & 14 & 1 &                                                  7.75    &   0.01    &   17.41     &      0.00       &     0.00  \NN
7 &LoggerHelper & Hibernate & 17 & 1 &                                                   3.73    &   0.01    &   87.50     &      0.00       &     33.33 \NN
8 &OAuthConfig & Scribe                             & 63 & 3 &                           3.21    &   0.05    &   91.14     &      100.00     &     100.00\NN
9 &OAuthRequest & Scribe                            & 36 & 2 &                           3.15    &   0.03    &   100.00    &      100.00     &     80.00 \NN
10&ParameterList & Scribe                           & 95 & 3 &                           3.19    &   0.06    &   99.60     &      96.83      &     81.99 \NN
11&Request & Scribe                                 & 213 & 2 &                          3.19    &   0.11    &   64.45     &      40.91      &     37.21 \NN
12&Response & Scribe                                & 72 & 2 &                           3.14    &   0.01    &   1.77      &      0.00       &     0.00  \NN
13&Token & Scribe                                   & 64 & 2 &                           3.17    &   0.08    &   100.00    &      94.64      &     90.91 \NN
14&Verifier & Scribe                                & 15 & 1 &                           3.13    &   0.01    &   100.00    &      0.00       &     50.00 \NN
15&ExceptionDiagnosis & Twitter4j                   & 73 & 4 &                           3.27    &   0.07    &   98.30     &      91.27      &     60.00 \NN
16&GeoQuery & Twitter4j                             & 126 & 2 &                          3.30    &   0.18    &   100.00    &      95.83      &     81.86 \NN
17&OEmbedRequest & Twitter4j                        & 147 & 2 &                          3.30    &   0.17    &   95.47     &      88.36      &     57.14 \NN
18&Paging & Twitter4j                               & 171 & 3 &                          3.24    &   0.14    &   93.91     &      91.27      &     68.76 \NN
19&TwitterBaseImpl & Twitter4j                      & 328 & 5 &                          6.73    &   0.04    &   9.16      &      6.16       &     3.54  \NN
20&TwitterException & Twitter4j                     & 237 & 4 &                          3.40    &   0.24    &   78.15     &      70.12      &     17.56 \NN
21&TwitterImpl & Twitter4j                          & 1187 & 3 &                         2.75    &   0.06    &   1.11      &      3.90       &     0.36  \NN
22&AsyncHttpClient & Async Http Client                    & 296 & 5 &                    9.02    &   0.00    &   0.00      &      0.00       &     0.00  \NN
23&AsyncHttpClientConfig & Async Http Client              & 621 & 2 &                    8.71    &   0.11    &   39.76     &      27.62      &     32.34 \NN
24&FluentCaseInsensitiveStringsMap & Async Http Client    & 292 & 4 &                    3.35    &   0.24    &   74.62     &      67.25      &     68.69 \NN
25&FluentStringsMap & Async Http Client                   & 240 & 4 &                    3.31    &   0.24    &   72.77     &      67.11      &     65.95 \NN
26&Realm & Async Http Client                              & 481 & 3 &                    3.43    &   0.28    &   92.98     &      64.65      &     70.09 \NN
27&RequestBuilderBase & Async Http Client                 & 544 & 6 &                    8.59    &   0.09    &   23.76     &      21.31      &     16.83 \NN
28&SimpleAsyncHttpClient & Async Http Client              & 610 & 3 &                    2.70    &   0.00    &   0.00      &      0.00       &     0.00  \NN
29&AttributeHelper & GData Java Client              & 344 & 4 &                          3.45    &   0.22    &   85.90     &      80.49      &     59.07 \NN
30&DateTime & GData Java Client                     & 264 & 5 &                          3.30    &   0.19    &   78.97     &      67.55      &     56.59 \NN
31&Kind & GData Java Client                         & 189 & 5 &                          3.36    &   0.06    &   53.31     &      43.83      &     46.27 \NN
32&Link & GData Java Client                         & 190 & 4 &                          3.79    &   0.17    &   62.98     &      59.46      &     36.43 \NN
33&OtherContent & GData Java Client                 & 179 & 4 &                          4.37    &   0.09    &   46.85     &      37.01      &     36.07 \NN
34&OutOfLineContent & GData Java Client             & 102 & 4 &                          4.13    &   0.15    &   83.55     &      81.12      &     67.26 \NN
35&Source & GData Java Client                       & 324 & 4 &                          3.47    &   0.23    &   42.59     &      30.84      &     33.04 \NN
36&CharMatcher & Guava                              & 824 & 4 &                          6.96    &   0.07    &   70.86     &      65.00      &     47.73 \NN
37&Joiner & Guava                                   & 222 & 3 &                          3.30    &   0.09    &   84.40     &      90.06      &     77.20 \NN
38&Objects & Guava                                  & 130 & 4 &                          3.24    &   0.12    &   96.20     &      79.37      &     96.14 \NN
39&Predicates & Guava                               & 379 & 3 &                          3.31    &   0.13    &   36.99     &      19.35      &     22.63 \NN
40&SmallCharMatcher & Guava                         & 92 & 4 &                           3.22    &   0.02    &   55.77     &      30.77      &     26.79 \NN
41&Splitter & Guava                                 & 266 & 4 &                          3.24    &   0.11    &   93.74     &      87.91      &     82.89 \NN
42&Suppliers & Guava                                & 172 & 4 &                          3.18    &   0.04    &   30.03     &      21.43      &     27.60 \NN
43&CategoryDescendantsIterator & JWPL               & 103 & 4 &                          3.18    &   0.01    &   8.24      &      0.00       &     0.00  \NN
44&CycleHandler & JWPL                              & 71 & 4 &                           3.09    &   0.01    &   21.41     &      20.71      &     27.73 \NN
45&Page & JWPL                                      & 351 & 4 &                          2.81    &   0.01    &   0.64      &      2.00       &     1.90  \NN
46&PageIterator & JWPL                              & 182 & 7 &                          3.14    &   0.05    &   46.55     &      40.82      &     34.23 \NN
47&PageQueryIterable & JWPL                         & 131 & 3 &                          2.62    &   0.03    &   14.93     &      7.31       &     0.00  \NN
48&Title & JWPL                                     & 79 & 2 &                           3.16    &   0.03    &   76.01     &      76.79      &     92.86 \NN
49&WikipediaInfo & JWPL                             & 222 & 6 &                          3.19    &   0.07    &   10.78     &      10.29      &     12.58 \NN
50&AbstractLoader & eclipse-cs                      & 77 & 3 &                           3.30    &   0.01    &   77.00     &      50.00      &     18.57 \NN
51&AnnotationUtility & eclipse-cs                   & 78 & 4 &                           3.24    &   0.10    &   58.11     &      54.29      &     46.75 \NN
52&AutomaticBean & eclipse-cs                       & 178 & 4 &                          4.45    &   0.02    &   49.45     &      29.59      &     13.29 \NN
53&FileContents & eclipse-cs                        & 161 & 4 &                          3.35    &   0.10    &   59.57     &      53.30      &     50.29 \NN
54&FileText & eclipse-cs                            & 184 & 4 &                          3.43    &   0.09    &   48.53     &      45.60      &     55.93 \NN
55&ScopeUtils & eclipse-cs                          & 189 & 4 &                          6.15    &   0.04    &   48.73     &      33.29      &     10.07 \NN
56&Utils & eclipse-cs                               & 160 & 4 &                          3.49    &   0.13    &   55.75     &      78.02      &     65.31 \NN
57&AbstractInstance & JMLL                          & 136 & 2 &                          3.69    &   0.08    &   87.06     &      67.86      &     68.08 \NN
58&Complex & JMLL                                   & 53 & 1 &                           3.18    &   0.08    &   100.00    &      0.00       &     47.93 \NN
59&DefaultDataset & JMLL                            & 152 & 3 &                          3.30    &   0.10    &   95.04     &      97.14      &     60.60 \NN
60&DenseInstance & JMLL                             & 155 & 3 &                          4.07    &   0.15    &   98.84     &      98.66      &     76.73 \NN
61&Fold & JMLL                                      & 201 & 2 &                          3.26    &   0.16    &   87.56     &      82.86      &     80.36 \NN
62&SparseInstance & JMLL                            & 191 & 3 &                          3.32    &   0.16    &   98.17     &      87.50      &     72.90 \NN
63&ARFFHandler & JMLL                               & 51 & 6 &                           3.15    &   0.01    &   0.00      &      0.00       &     0.00  \LL
%\bottomrule
%\end{tabular}
%\end{table}
%\caption{The CUTs that constitute the benchmark}\label{benchmark-classes}
} }

Coverage results achieved by \EVOSUITE are listed in Table~\ref{table:results}. On the 63 classes of the benchmark used in the competition, \EVOSUITE
produced an average instruction coverage of 59.9\%, average branch
coverage of 48.63\%, and average mutation score of 43.8\%. These
results are in line with our expectations based on recent
experimentation~\cite{FrA12b}.

For each class it produced on average 12.3 test cases. For the entire
benchmark of 63 classes, \EVOSUITE took on average 4 hours for test
generation, which means that \EVOSUITE spent an additional 51 seconds
per class on the up-front analysis of the classpath as well as the
post-processing after the search.

In the following discussion, we
%As a detailed discussion of the results on all 63 classes is not
%possible due to space restrictions, we 
focus on some interesting cases where \EVOSUITE achieved no or very
little ($<2\%$) coverage or mutation score.

{\bf Trivial classes: } Class 1 ({\it SearchException}) is a trivial
class with only 15 lines of code, and it seems surprising that
\EVOSUITE achieved 0\% coverage here. The reason for this bad result
is that the class consists of four constructors that do nothing but
calling the constructor of the superclass, and as a result there are
no statements for which \EVOSUITE produced any mutants, and there are
no branching instructions. While the standard branch coverage
criterion in \EVOSUITE also enforces that each method is executed at
least once, the weak mutation testing criterion we used in the
competition did not do so at the time of the competition, and so
\EVOSUITE produced no tests.

{\bf Classpath dependent behavior: } Class 2 ({\it Version}) is
another trivial class, yet the reason for the bad result do not lie in
\EVOSUITE, but rather the frontend of \EVOSUITE we built to interface
with the competition infrastructure. When setting up the classpath for
\EVOSUITE, our competition frontend unnecessarily included the source
directory of the unit under test in the classpath. In the competition
setup there are compiled versions of the classes in the source
directory as well as a dedicated target path; unfortunately, they
differ: Method {\it getVersionString} returns ``[WORKING]'' in the
class in the source directory, whereas the deployed version of the
class is changed to return ``[0.4.4-SNAPSHOT]''. As \EVOSUITE's
assertions were expecting ``[WORKING]'', resulting tests failed and
were not considered for mutation analysis.

{\bf Nondeterministic code: } Class 30 ({\it DateTime}) makes use of
the current system time, such that automatically generated assertions
may refer to the time of test execution. If this happens, then the
tests will fail, and will not be considered for mutation
analysis. This is a known issue, and \EVOSUITE overcomes it by using
bytecode instrumentation that replaces nondeterministic
calls. However, as the PIT mutation testing tool used in the
competition is not able to handle JUnit tests with this kind of
instrumentation, we had to deactivate it. \EVOSUITE does compile and
execute tests as a last step to comment out any failing assertions;
yet in this case there were assertions dependent on the seconds of the
current time, and so this verification step was too quick to notice
the failing assertions.

{\bf Environmental dependencies: } One of the largest problems in unit
testing remains the handling of environmental dependencies. For
example, most classes in the JWPL project (43--49) depend on a valid
instance of a {\it Wikipedia} class, which in turn depends on a valid
{\it DatabaseConfiguration}. As another example, class 63
(ARFFHandler) consists of only one method (and one wrapper for that
method) that receives a file as parameter, and then tries to parse
this file.

%12. Response Scribe: 54 mutants, but must be some other problem.

{\bf Complex classes: } The benchmark included large classes, most
notably class 21 ({\it TwitterImpl}): This class has 1,187 lines of
code, and \EVOSUITE produced 2,453 mutants for it. That on its own
would not be a challenge for \EVOSUITE; however, test execution on
this class turns out to be very slow, and can lead to excessive memory
consumption. To prevent out of memory exceptions from occurring and
crashing \EVOSUITE, \EVOSUITE cancels test generation when the Java
garbage collector fails to free sufficient memory. This happened not
only in this class, but also several others (19, 23, 43, 44, 45, 47,
49). Running \EVOSUITE with more memory would have likely resulted in
higher coverage on these classes.



%22. AsyncHttpClient: Slow (few statements executed) and timeout -
%master killed client in most cases.

%28. SimpleAsyncHttpClient: Error during setup
%(java.lang.ExceptionInInitializerError during call tree
%initialisation).

%43: CategoryDescendantsIterator: Compile errors - complains that
%de.fau.cs.osr.utils is missing. The test does not actually include
%that, but the SUT! Is the classpath correct?

%44 CycleHandler - same problem

%45 Page - same problem

%46 PageIterator - same problem

%47 PageQueryIterable - same problem

%49 WikipediaInfo - same problem

%63 ARFFHandler: No performance problem, but EvoSuite can't produce
%tests, so probably some error in generating an object.

\section{Conclusions}

With an overall score of 205.26, \EVOSUITE achieved the highest score
of all tools in the competition. The score calculated for manual
testing is 210.45 --- a very close call. This means that \EVOSUITE is
achieving almost human-competitive results in terms of the
effectiveness of the resulting test suites. Potentially, just by
fixing some of the errors in \EVOSUITE or its competition frontend the
resulting score could be higher than 210, even without adding new
features. We are confident that some of the experimental features will
further increase this once the required level of robustness has been
achieved.

\section*{Acknowledgements}
This project has been funded by a Google
Focused Research Award on ``Test Amplification'' and the Norwegian
Research Council.

\bibliographystyle{splncs03}
\bibliography{../papers}


\end{document}


