\documentclass[sigconf]{acmart}
\usepackage{amsfonts}
\usepackage[latin1]{inputenc}
\usepackage[english]{babel}
\usepackage{listings}
\usepackage{algorithmic}
\usepackage{float}
%\usepackage[numbers,sort&compress,square]{natbib}
\usepackage{graphicx}
\usepackage{booktabs}
\usepackage{subfigure}
%\usepackage{hyperref}
\usepackage{color}
%\usepackage[usenames,dvipsnames,]{xcolor}
%\usepackage{soul}
\usepackage{xspace}
\usepackage{boxedminipage}
\usepackage{alltt}
\usepackage{multirow}
\usepackage{paralist}
\usepackage{amsmath}
\usepackage{balance}
\definecolor{light-gray}{gray}{0.90}


\newcommand{\EVOSUITE}{{\sc EvoSuite}\xspace}
\newcommand{\RANDOOP}{{\sc Randoop}\xspace}
\newcommand{\MUTEST}{{\sc $\mu$Test}\xspace}
\newcommand{\CS}{{\sc SF100}\xspace}
\newcommand{\TOTALPOINTS}{{255.43}\xspace}


%%
%% \BibTeX command to typeset BibTeX logo in the docs
\AtBeginDocument{%
  \providecommand\BibTeX{{%
    \normalfont B\kern-0.5em{\scshape i\kern-0.25em b}\kern-0.8em\TeX}}}

%% Rights management information.  This information is sent to you
%% when you complete the rights form.  These commands have SAMPLE
%% values in them; it is your responsibility as an author to replace
%% the commands and values with those provided to you when you
%% complete the rights form.
\setcopyright{acmcopyright}
\copyrightyear{2018}
\acmYear{2018}
\acmDOI{10.1145/1122445.1122456}

%% These commands are for a PROCEEDINGS abstract or paper.
\acmConference[SBST '20]{SBST'20: IEEE/ACM 13th International Workshop on Search-Based Software Testing}{October 2020}{Seoul, South Korea}
\acmBooktitle{SBST'20: IEEE/ACM 13th International Workshop on Search-Based Software Testing,
  October 2020, Seoul, South Korea}
\acmPrice{15.00}
\acmISBN{978-1-4503-9999-9/18/06}


%%
%% Submission ID.
%% Use this when submitting an article to a sponsored event. You'll
%% receive a unique submission ID from the organizers
%% of the event, and this ID should be used as the parameter to this command.
%%\acmSubmissionID{123-A56-BU3}

%%
%% The majority of ACM publications use numbered citations and
%% references.  The command \citestyle{authoryear} switches to the
%% "author year" style.
%%
%% If you are preparing content for an event
%% sponsored by ACM SIGGRAPH, you must use the "author year" style of
%% citations and references.
%% Uncommenting
%% the next command will enable that style.
%%\citestyle{acmauthoryear}

%%
%% end of the preamble, start of the body of the document source.

\begin{document}

%%
%% The "title" command has an optional parameter,
%% allowing the author to define a "short title" to be used in page headers.
\title{\EVOSUITE at the SBST 2020 Tool Competition}

%%
%% The "author" command and its associated commands are used to define
%% the authors and their affiliations.
%% Of note is the shared affiliation of the first two authors, and the
%% "authornote" and "authornotemark" commands
%% used to denote shared contribution to the research.
\author{Annibale Panichella}
%\orcid{0000-0002-5385-7695}
%\author{G.K.M. Tobin}
%\authornotemark[1]
%\email{webmaster@marysville-ohio.com}
\affiliation{%
  \institution{Delft University of Technology}
  \state{The Netherlands}
}
\email{a.panichella@tudelft.nl}

\author{Jos\'e Campos}
%\email{a.panichella@tudelft.nl}
%\orcid{0000-0002-5385-7695}
%\author{G.K.M. Tobin}
%\authornotemark[1]
\affiliation{%
  \institution{LASIGE, Faculdade de Ci\^encias \\ Universidade de Lisboa}
  \state{Portugal}
}
%\email{webmaster@marysville-ohio.com}

\author{Gordon Fraser}
\affiliation{%
  \institution{Chair of Software Engineering II, University of Passau}
  \city{Passau}
  \country{Germany}
}
\email{gordon.fraser@uni-passau.de}

%\begin{teaserfigure}

%\end{teaserfigure}

%%
%% By default, the full list of authors will be used in the page
%% headers. Often, this list is too long, and will overlap
%% other information printed in the page headers. This command allows
%% the author to define a more concise list
%% of authors' names for this purpose.
%\renewcommand{\shortauthors}{Anonymous et al.}

%%
%% The abstract is a short summary of the work to be presented in the
%% article.
\begin{abstract}
TBD
\end{abstract}

\maketitle

\section{Introduction}

\section{\EVOSUITE}

\section{Tool Setup}

\section{Benchmark Results}

\section{Conclusion}

\bibliographystyle{ACM-Reference-Format}
\bibliography{papers}






\end{document}
