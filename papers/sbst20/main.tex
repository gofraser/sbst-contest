\documentclass[sigconf]{acmart}
\usepackage{amsfonts}
\usepackage[latin1]{inputenc}
\usepackage[english]{babel}
\usepackage{listings}
\usepackage{algorithmic}
\usepackage{float}
%\usepackage[numbers,sort&compress,square]{natbib}
\usepackage{graphicx}
\usepackage{booktabs}
\usepackage{subfigure}
%\usepackage{hyperref}
\usepackage{xspace}
\usepackage{xcolor,colortbl}
%\usepackage[usenames,dvipsnames,]{xcolor}
%\usepackage{soul}
%\usepackage{boxedminipage}
%\usepackage{alltt}
\usepackage{multirow}
%\usepackage{paralist}
%\usepackage{amsmath}
\usepackage{balance}
\definecolor{light-gray}{gray}{0.90}


\newcommand{\EVOSUITE}{{\sc EvoSuite}\xspace}
\newcommand{\RANDOOP}{{\sc Randoop}\xspace}
\newcommand{\MUTEST}{{\sc $\mu$Test}\xspace}
\newcommand{\CS}{{\sc SF100}\xspace}
\newcommand{\TOTALPOINTS}{{255.43}\xspace}


%%
%% \BibTeX command to typeset BibTeX logo in the docs
\AtBeginDocument{%
  \providecommand\BibTeX{{%
    \normalfont B\kern-0.5em{\scshape i\kern-0.25em b}\kern-0.8em\TeX}}}

%% Rights management information.  This information is sent to you
%% when you complete the rights form.  These commands have SAMPLE
%% values in them; it is your responsibility as an author to replace
%% the commands and values with those provided to you when you
%% complete the rights form.
\setcopyright{acmcopyright}
\copyrightyear{2018}
\acmYear{2018}
\acmDOI{10.1145/1122445.1122456}

%% These commands are for a PROCEEDINGS abstract or paper.
\acmConference[SBST '20]{SBST'20: IEEE/ACM 13th International Workshop on Search-Based Software Testing}{October 2020}{Seoul, South Korea}
\acmBooktitle{SBST'20: IEEE/ACM 13th International Workshop on Search-Based Software Testing,
  October 2020, Seoul, South Korea}
\acmPrice{15.00}
\acmISBN{978-1-4503-9999-9/18/06}


%%
%% Submission ID.
%% Use this when submitting an article to a sponsored event. You'll
%% receive a unique submission ID from the organizers
%% of the event, and this ID should be used as the parameter to this command.
%%\acmSubmissionID{123-A56-BU3}

%%
%% The majority of ACM publications use numbered citations and
%% references.  The command \citestyle{authoryear} switches to the
%% "author year" style.
%%
%% If you are preparing content for an event
%% sponsored by ACM SIGGRAPH, you must use the "author year" style of
%% citations and references.
%% Uncommenting
%% the next command will enable that style.
%%\citestyle{acmauthoryear}

%%
%% end of the preamble, start of the body of the document source.

\begin{document}

%%
%% The "title" command has an optional parameter,
%% allowing the author to define a "short title" to be used in page headers.
\title{\EVOSUITE at the SBST 2020 Tool Competition}

%%
%% The "author" command and its associated commands are used to define
%% the authors and their affiliations.
%% Of note is the shared affiliation of the first two authors, and the
%% "authornote" and "authornotemark" commands
%% used to denote shared contribution to the research.
\author{Annibale Panichella}
%\orcid{0000-0002-5385-7695}
%\author{G.K.M. Tobin}
%\authornotemark[1]
%\email{webmaster@marysville-ohio.com}
\affiliation{%
  \institution{Delft University of Technology}
  \state{The Netherlands}
}
\email{a.panichella@tudelft.nl}

\author{Jos\'e Campos}
%\email{a.panichella@tudelft.nl}
%\orcid{0000-0002-5385-7695}
%\author{G.K.M. Tobin}
%\authornotemark[1]
\affiliation{%
  \institution{LASIGE, Faculdade de Ci\^encias \\ Universidade de Lisboa}
  \city{Lisboa}
  \country{Portugal}
}
\email{jcmcampos@fc.ul.pt}

\author{Gordon Fraser}
\affiliation{%
  \institution{Chair of Software Engineering II, University of Passau}
  \city{Passau}
  \country{Germany}
}
\email{gordon.fraser@uni-passau.de}

%\begin{teaserfigure}

%\end{teaserfigure}

%%
%% By default, the full list of authors will be used in the page
%% headers. Often, this list is too long, and will overlap
%% other information printed in the page headers. This command allows
%% the author to define a more concise list
%% of authors' names for this purpose.
%\renewcommand{\shortauthors}{Anonymous et al.}

%%
%% The abstract is a short summary of the work to be presented in the
%% article.
\begin{abstract}
TBD
\end{abstract}

\maketitle

\section{Introduction}

\section{\EVOSUITE}

\begin{table}[!h]
\renewcommand{\arraystretch}{1.3}
\caption{Classification of the \EVOSUITE unit test generation
  tool}\label{tool-description}
\resizebox{1.0\columnwidth}{!}{  
\begin{tabular}{|l|p{5cm}|}
  \hline
  \multicolumn{2}{|l|}{Prerequisites} \\
  \hline
  Static or dynamic &  Dynamic testing at the Java class level\\
  Software Type &  Java classes\\
  Lifecycle phase&  Unit testing for Java programs\\
  Environment&  All Java development environments \\
  Knowledge required & JUnit unit testing for Java\\
  Experience required &  Basic unit testing knowledge\\
 \hline
  \multicolumn{2}{|l|}{Input and Output of the tool} \\
  \hline
 Input & Bytecode of the target class and dependencies \\
\hline
Output&  JUnit 4 test cases\\
 
  \hline
  \multicolumn{2}{|l|}{Operation} \\
  \hline
  Interaction &  Through the command line, and plugins for IntelliJ, Maven and Eclipse\\
  User guidance &  Manual verification of assertions for functional faults\\
  Source of information &  http://www.evosuite.org \\
  Maturity&  Mature research prototype, under development\\
  Technology behind the tool & Search-based testing / many-objective optimization \\
\hline
  \multicolumn{2}{|l|}{Obtaining the tool and information} \\
  \hline
License & Lesser GPL V.3\\
Cost & Open source\\
Support & None \\
\hline
\hline
  \multicolumn{2}{|l|}{Does there exist empirical evidence about} \\
  \hline
  Effectiveness and Scalability & See~\cite{GoA_TSE12,fraser2014large} \\
%Completeness & \\
%Effectiveness & \\
%Efficiency & \\
%Defect types & \\
%Scalability & \\
%Comprehensibility & \\
%Learnability & \\
%Subjective satisfaction & \\
%Other & \\
\hline
\end{tabular}\vspace{-1em}
}
\end{table}


\section{Tool Setup}

\section{Benchmark Results}


\begin{table*}[t]
  \centering
  \caption{\label{table:results}Detailed results of \EVOSUITE on the
    SBST benchmark classes.}
\vspace{-1em}
\resizebox{0.8\textwidth}{!}{  
\begin{tabular}{ ll rr rr rr}\toprule 
 \multirow{2}{1in}{Benchmark} & \multirow{2}{1in}{Java Class} &  \multicolumn{2}{c}{Line Coverage} &  \multicolumn{2}{c}{Branch Coverage} &  \multicolumn{2}{c}{Mutation Score}\\\cmidrule(lr){3-4}\cmidrule(lr){5-6} \cmidrule(lr){7-8}
 & & 60s & 180s & 60s & 180s & 60s & 180s \\ 
\midrule 
FESCAR-10  &  com.alibaba.fescar.core.model.BranchType & 80.0\% & 90.0\% & 80.0\% & 90.0\% & 80.0\% & 90.0\%\\ 
FESCAR-12  &  com.alibaba.fescar.core.rpc.netty.RpcServerHandler & 100.0\% & 100.0\% & 87.5\% & 87.5\% & 100.0\% & 100.0\%\\ 
FESCAR-13  &  com.alibaba.fescar.core.exception.TransactionExceptionCode & 100.0\% & 100.0\% & 100.0\% & 100.0\% & 100.0\% & 100.0\%\\ 
FESCAR-15  &  com.alibaba.fescar.core.rpc.netty.RpcServer & 0.8\% & 0.7\% & \cellcolor{light-gray} \textcolor{black}{0.0\%} & \cellcolor{light-gray} \textcolor{black}{0.0\%} & \cellcolor{light-gray} \textcolor{black}{0.0\%} & \cellcolor{light-gray} \textcolor{black}{0.0\%}\\ 
FESCAR-17  &  com.alibaba.fescar.core.protocol.transaction.GlobalBeginResponse & 99.4\% & 99.4\% & 100.0\% & 100.0\% & 90.0\% & 90.0\%\\ 
FESCAR-2  &  com.alibaba.fescar.core.service.ServiceManagerStaticConfigImpl & 20.5\% & 25.8\% & \cellcolor{light-gray} \textcolor{black}{0.0\%} & \cellcolor{light-gray} \textcolor{black}{0.0\%} & \cellcolor{light-gray} \textcolor{black}{0.0\%} & \cellcolor{light-gray} \textcolor{black}{0.0\%}\\ 
FESCAR-23  &  com.alibaba.fescar.core.protocol.MergeResultMessage & 90.5\% & 60.5\% & 76.4\% & 50.0\% & \cellcolor{light-gray} \textcolor{black}{0.0\%} & \cellcolor{light-gray} \textcolor{black}{0.0\%}\\ 
FESCAR-25  &  com.alibaba.fescar.core.rpc.netty.RmMessageListener & 46.9\% & 37.5\% & 62.5\% & 48.8\% & 22.2\% & 17.8\%\\ 
FESCAR-28  &  com.alibaba.fescar.core.rpc.ClientType & 90.0\% & 100.0\% & 90.0\% & 100.0\% & 90.0\% & 100.0\%\\ 
FESCAR-32  &  com.alibaba.fescar.core.protocol.transaction.BranchRegisterRequest & 97.7\% & 89.2\% & 94.4\% & 87.5\% & 95.2\% & 78.3\%\\ 
FESCAR-33  &  com.alibaba.fescar.core.model.GlobalStatus & 100.0\% & 100.0\% & 100.0\% & 100.0\% & 100.0\% & 100.0\%\\ 
FESCAR-34  &  com.alibaba.fescar.core.protocol.ResultCode & 90.0\% & 100.0\% & 90.0\% & 100.0\% & 90.0\% & 100.0\%\\ 
FESCAR-37  &  com.alibaba.fescar.core.rpc.RpcContext & 92.4\% & 94.6\% & 86.8\% & 91.2\% & \cellcolor{light-gray} \textcolor{black}{0.0\%} & \cellcolor{light-gray} \textcolor{black}{0.0\%}\\ 
FESCAR-41  &  com.alibaba.fescar.core.rpc.netty.RmRpcClient & 1.7\% & 1.7\% & 2.0\% & 2.0\% & \cellcolor{light-gray} \textcolor{black}{0.0\%} & 2.4\%\\ 
FESCAR-42  &  com.alibaba.fescar.core.rpc.DefaultServerMessageListenerImpl & 24.3\% & 42.6\% & 11.8\% & 27.1\% & 12.1\% & 25.4\%\\ 
FESCAR-5  &  com.alibaba.fescar.core.protocol.MessageFuture & 98.6\% & 99.1\% & 96.0\% & 98.0\% & 99.2\% & 100.0\%\\ 
FESCAR-6  &  com.alibaba.fescar.core.rpc.netty.TmRpcClient & 3.4\% & 3.4\% & 2.7\% & 2.7\% & \cellcolor{light-gray} \textcolor{black}{0.0\%} & 2.7\%\\ 
FESCAR-7  &  com.alibaba.fescar.core.rpc.netty.MessageCodecHandler & 76.1\% & 78.2\% & 73.3\% & 77.2\% & \cellcolor{light-gray} \textcolor{black}{0.0\%} & \cellcolor{light-gray} \textcolor{black}{0.0\%}\\ 
FESCAR-8  &  com.alibaba.fescar.core.rpc.netty.NettyPoolableFactory & 57.3\% & 62.0\% & 50.8\% & 57.5\% & \cellcolor{light-gray} \textcolor{black}{0.0\%} & \cellcolor{light-gray} \textcolor{black}{0.0\%}\\ 
FESCAR-9  &  com.alibaba.fescar.core.protocol.transaction.GlobalBeginRequest & 99.0\% & 98.3\% & 100.0\% & 100.0\% & 99.1\% & 98.2\%\\ 
GUAVA-102  &  com.google.common.collect.LinkedListMultimap & 29.4\% & 32.3\% & 12.9\% & 11.6\% & 19.2\% & 14.8\%\\ 
GUAVA-110  &  com.google.common.collect.LexicographicalOrdering & 3.0\% & 22.2\% & \cellcolor{light-gray} \textcolor{black}{0.0\%} & 7.5\% & 0.6\% & 15.0\%\\ 
GUAVA-128  &  com.google.common.base.Throwables & 75.1\% & 25.0\% & 75.8\% & 25.3\% & 81.0\% & 26.8\%\\ 
GUAVA-129  &  com.google.common.collect.SparseImmutableTable & 31.9\% & 35.8\% & 37.5\% & 42.5\% & 35.0\% & 43.8\%\\ 
GUAVA-159  &  com.google.common.primitives.ParseRequest & 100.0\% & 100.0\% & 100.0\% & 100.0\% & 50.0\% & 50.0\%\\ 
GUAVA-169  &  com.google.common.math.LongMath & 96.2\% & 86.7\% & 94.2\% & 85.3\% & 99.2\% & 89.3\%\\ 
GUAVA-177  &  com.google.common.primitives.Doubles & 98.7\% & 98.5\% & 99.3\% & 99.3\% & 100.0\% & 100.0\%\\ 
GUAVA-181  &  com.google.common.primitives.SignedBytes & 100.0\% & 100.0\% & 100.0\% & 100.0\% & 100.0\% & 100.0\%\\ 
GUAVA-184  &  com.google.thirdparty.publicsuffix.PublicSuffixType & 100.0\% & 100.0\% & 100.0\% & 100.0\% & 100.0\% & 100.0\%\\ 
GUAVA-196  &  com.google.common.io.Closeables & 71.5\% & 70.0\% & 77.5\% & 75.0\% & 88.0\% & 88.0\%\\ 
GUAVA-2  &  com.google.common.collect.MinMaxPriorityQueue & 13.9\% & 22.5\% & 6.4\% & 11.1\% & 16.5\% & 19.2\%\\ 
GUAVA-206  &  com.google.common.collect.ImmutableEnumSet & 25.4\% & 26.1\% & 23.6\% & 24.5\% & 7.1\% & 7.6\%\\ 
GUAVA-212  &  com.google.common.net.MediaType & 92.6\% & 94.3\% & 77.6\% & 83.0\% & \cellcolor{light-gray} \textcolor{black}{0.0\%} & \cellcolor{light-gray} \textcolor{black}{0.0\%}\\ 
GUAVA-22  &  com.google.common.graph.Graphs & 53.9\% & 49.7\% & 51.8\% & 47.3\% & \cellcolor{light-gray} \textcolor{black}{0.0\%} & \cellcolor{light-gray} \textcolor{black}{0.0\%}\\ 
GUAVA-224  &  com.google.common.primitives.UnsignedLongs & 99.3\% & 89.6\% & 100.0\% & 90.0\% & 100.0\% & 90.0\%\\ 
GUAVA-240  &  com.google.common.collect.FilteredMultimapValues & 12.3\% & 22.7\% & \cellcolor{light-gray} \textcolor{black}{0.0\%} & 5.0\% & \cellcolor{light-gray} \textcolor{black}{0.0\%} & \cellcolor{light-gray} \textcolor{black}{0.0\%}\\ 
GUAVA-39  &  com.google.common.collect.TreeMultiset & 30.2\% & 43.1\% & 18.6\% & 27.9\% & 19.5\% & 31.3\%\\ 
GUAVA-47  &  com.google.common.collect.FilteredEntryMultimap & 2.6\% & 11.3\% & \cellcolor{light-gray} \textcolor{black}{0.0\%} & 0.7\% & \cellcolor{light-gray} \textcolor{black}{0.0\%} & 0.4\%\\ 
GUAVA-90  &  com.google.common.io.FileBackedOutputStream & 98.9\% & 89.6\% & 98.1\% & 90.0\% & 98.0\% & 89.3\%\\ 
GUAVA-95  &  com.google.common.collect.ComparatorOrdering & 27.5\% & 51.7\% & 12.5\% & 30.0\% & 18.8\% & 31.2\%\\ 
PDFBOX-117  &  org.apache.pdfbox.filter.Predictor & 89.0\% & 93.5\% & 83.9\% & 91.0\% & \cellcolor{light-gray} \textcolor{black}{0.0\%} & 28.6\%\\ 
PDFBOX-127  &  org.apache.pdfbox.pdfparser.PDFObjectStreamParser & 57.5\% & 65.6\% & 37.1\% & 43.6\% & 44.4\% & 50.6\%\\ 
PDFBOX-130  &  org.apache.pdfbox.pdmodel.interactive.digitalsignature.visible.PDVisibleSignDesigner & 7.1\% & 14.3\% & 1.7\% & 1.7\% & 1.5\% & 2.5\%\\ 
PDFBOX-157  &  org.apache.pdfbox.pdmodel.font.PDType1Font & 2.1\% & \cellcolor{light-gray} \textcolor{black}{0.0\%} & 0.4\% & \cellcolor{light-gray} \textcolor{black}{0.0\%} & \cellcolor{light-gray} \textcolor{black}{0.0\%} & \cellcolor{light-gray} \textcolor{black}{0.0\%}\\ 
PDFBOX-198  &  org.apache.pdfbox.pdmodel.fdf.FDFAnnotationLine & 66.4\% & 66.5\% & 32.4\% & 32.7\% & 5.5\% & \cellcolor{light-gray} \textcolor{black}{0.0\%}\\ 
PDFBOX-214  &  org.apache.pdfbox.pdfparser.EndstreamOutputStream & 99.5\% & 90.0\% & 99.2\% & 90.0\% & 48.0\% & 40.0\%\\ 
PDFBOX-22  &  org.apache.pdfbox.pdmodel.fdf.FDFAnnotationCaret & 63.9\% & 63.9\% & 64.3\% & 64.3\% & 10.5\% & 31.4\%\\ 
PDFBOX-220  &  org.apache.pdfbox.filter.JPXFilter & 32.7\% & 32.7\% & 7.7\% & 7.3\% & \cellcolor{light-gray} \textcolor{black}{0.0\%} & \cellcolor{light-gray} \textcolor{black}{0.0\%}\\ 
PDFBOX-229  &  org.apache.pdfbox.util.XMLUtil & 62.4\% & 69.6\% & 52.5\% & 60.0\% & 10.7\% & 13.6\%\\ 
PDFBOX-234  &  org.apache.pdfbox.pdmodel.interactive.action.PDActionSound & 97.7\% & 96.7\% & 88.9\% & 87.8\% & \cellcolor{light-gray} \textcolor{black}{0.0\%} & 20.0\%\\ 
PDFBOX-235  &  org.apache.pdfbox.pdmodel.font.PDTrueTypeFontEmbedder & \cellcolor{light-gray} \textcolor{black}{0.0\%} & \cellcolor{light-gray} \textcolor{black}{0.0\%} & \cellcolor{light-gray} \textcolor{black}{0.0\%} & \cellcolor{light-gray} \textcolor{black}{0.0\%} & \cellcolor{light-gray} \textcolor{black}{0.0\%} & \cellcolor{light-gray} \textcolor{black}{0.0\%}\\ 
PDFBOX-26  &  org.apache.pdfbox.pdmodel.encryption.SecurityProvider & 55.8\% & 56.8\% & 100.0\% & 100.0\% & 100.0\% & 90.0\%\\ 
PDFBOX-265  &  org.apache.pdfbox.pdmodel.font.PDType3Font & 62.4\% & 70.2\% & 42.3\% & 52.0\% & \cellcolor{light-gray} \textcolor{black}{0.0\%} & \cellcolor{light-gray} \textcolor{black}{0.0\%}\\ 
PDFBOX-278  &  org.apache.pdfbox.pdfwriter.ContentStreamWriter & 96.8\% & 98.3\% & 96.7\% & 96.3\% & \cellcolor{light-gray} \textcolor{black}{0.0\%} & \cellcolor{light-gray} \textcolor{black}{0.0\%}\\ 
PDFBOX-285  &  org.apache.pdfbox.pdmodel.interactive.digitalsignature.PDSignature & 98.9\% & 99.7\% & 89.5\% & 95.5\% & \cellcolor{light-gray} \textcolor{black}{0.0\%} & \cellcolor{light-gray} \textcolor{black}{0.0\%}\\ 
PDFBOX-40  &  org.apache.pdfbox.pdmodel.font.PDCIDFontType2 & 57.2\% & 54.9\% & 45.1\% & 46.6\% & \cellcolor{light-gray} \textcolor{black}{0.0\%} & \cellcolor{light-gray} \textcolor{black}{0.0\%}\\ 
PDFBOX-62  &  org.apache.pdfbox.rendering.PageDrawer & 2.3\% & 6.8\% & 1.2\% & 4.2\% & \cellcolor{light-gray} \textcolor{black}{0.0\%} & \cellcolor{light-gray} \textcolor{black}{0.0\%}\\ 
PDFBOX-8  &  org.apache.pdfbox.pdmodel.font.FileSystemFontProvider & 45.2\% & 48.4\% & 34.2\% & 35.8\% & 41.9\% & 52.2\%\\ 
PDFBOX-83  &  org.apache.pdfbox.contentstream.operator.text.SetTextRenderingMode & 89.3\% & 85.7\% & 92.5\% & 100.0\% & 82.5\% & 87.5\%\\ 
PDFBOX-91  &  org.apache.pdfbox.pdmodel.documentinterchange.taggedpdf.PDArtifactMarkedContent & 91.6\% & 97.9\% & 71.2\% & 92.5\% & \cellcolor{light-gray} \textcolor{black}{0.0\%} & \cellcolor{light-gray} \textcolor{black}{0.0\%}\\ 
SPOON-105  &  spoon.support.compiler.jdt.PositionBuilder & 9.6\% & 5.5\% & 7.8\% & 3.9\% & \cellcolor{light-gray} \textcolor{black}{0.0\%} & \cellcolor{light-gray} \textcolor{black}{0.0\%}\\ 
SPOON-155  &  spoon.reflect.visitor.filter.AllMethodsSameSignatureFunction & 13.0\% & 12.7\% & \cellcolor{light-gray} \textcolor{black}{0.0\%} & 1.2\% & 0.7\% & 3.2\%\\ 
SPOON-16  &  spoon.reflect.path.CtElementPathBuilder & 15.9\% & 16.1\% & 8.0\% & 9.0\% & 10.3\% & 6.4\%\\ 
SPOON-169  &  spoon.reflect.visitor.ImportScannerImpl & 1.2\% & 10.6\% & 0.1\% & 4.7\% & \cellcolor{light-gray} \textcolor{black}{0.0\%} & 1.3\%\\ 
SPOON-20  &  spoon.support.reflect.reference.CtLocalVariableReferenceImpl & 30.0\% & 38.6\% & 14.0\% & 18.0\% & 3.3\% & 13.3\%\\ 
SPOON-211  &  spoon.reflect.path.impl.CtRolePathElement & 16.3\% & 18.3\% & 6.2\% & 10.3\% & 6.2\% & 11.2\%\\ 
SPOON-25  &  spoon.pattern.internal.ValueConvertorImpl & 3.0\% & 7.1\% & 1.2\% & 3.1\% & 0.7\% & 4.3\%\\ 
SPOON-253  &  spoon.pattern.internal.parameter.MapParameterInfo & 76.8\% & 73.9\% & 72.5\% & 73.8\% & \cellcolor{light-gray} \textcolor{black}{0.0\%} & \cellcolor{light-gray} \textcolor{black}{0.0\%}\\ 
SPOON-32  &  spoon.MavenLauncher & 27.0\% & 30.0\% & 11.2\% & 12.5\% & 6.0\% & 6.7\%\\ 
SPOON-65  &  spoon.support.DefaultCoreFactory & 10.7\% & 9.7\% & 5.9\% & 8.9\% & 0.1\% & \cellcolor{light-gray} \textcolor{black}{0.0\%}\\ 
\midrule 
Average &  &  55.9\% &  57.0\% &  50.8\% &  51.7\% &  32.6\% &  33.8\%\\ 
\bottomrule 
\end{tabular} 
}	
\end{table*}

%% FESCAR-15 0% coverage 0% mutation score

\subsubsection*{FESCAR-15} According to the \texttt{transcript.csv} file, it
seems that the generated test cases do cover some lines of code but no
branches/conditions.  Looking at the \texttt{logs.log}, \EVOSUITE also reported
0\% coverage, however there are not any issues in any log file.

%% FESCAR-2 0% coverage 0% mutation score

\subsubsection*{FESCAR-2} No clue.  \EVOSUITE does report some coverage at the
end of the \texttt{logs.log} file, but somehow (and according to JaCoCo) the
generated test suite does not cover any branch.

%% FESCAR-37 86.8% coverage 0% mutation score

\subsubsection*{FESCAR-37} The classpath is incomplete as there is a
\texttt{java.lang.ClassNotFoundException: io.netty.channel.Channel}.

%% GUAVA-110 0% coverage 0.6% mutation score

\subsubsection*{GUAVA-110} No clue.

%% FESCAR-7 73.3% coverage 0% mutation score
%% GUAVA-212 77.6% coverage 0% mutation score
%% GUAVA-22 51.8% coverage 0% mutation score
%% PDFBOX-265 51.8% coverage 0% mutation score

\subsubsection*{GUAVA-212, GUAVA-22, PDFBOX-265, PDFBOX-278, PDFBOX-285,
PDFBOX-40, PDFBOX-91} The mutation analysis failed due to a
\texttt{java.util.concurrent.ExecutionException} thrown by the experimental
infrastructure.

%% PDFBOX-62 1.2% coverage 0% mutation score

\subsubsection*{PDFBOX-62} \EVOSUITE crashed due to some Mocking stuff.

%% SPOON

\subsubsection*{SPOON} There is no \texttt{logs.log} file to check \EVOSUITE's
log messages.  There is any test suite for any SPOON
class.  The reason might be due to
\texttt{java.io.IOException: Not a valid directory name: /var/benchmarks/projects/spoon-core-7.2.0/src/main/java}.
in the \texttt{log\_detailed.txt} file.  Also, all \texttt{transcript.csv} are
empty.  Thus, it seems the experimental infrastructure do not even attempt to
generate test cases for any SPOON class.  Question: how did \EVOSUITE achieved
a 72.5\% coverage for, e.g., SPOON-253 if there is no test suite or even a valid
attempt?


\section{Conclusions}

\bibliographystyle{ACM-Reference-Format}
\bibliography{papers}






\end{document}
